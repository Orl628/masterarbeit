\section{Verlinde Bundles on Pencils of Quartics}

%\begin{definition} A \emph{line} is an embedding $l\colon \mathbb{P}^1 \to X$ into a projective scheme $X$ such that $l^*\mathcal{O}(1) = \mathcal{O}(1)$.
%\end{definition}

\newcommand{\schemeofquartics}{\abs{\mc{O}(4)}}

The thesis \cite{hemminghaus-verlinde-bundles} studies Verlinde bundles for
families of polarized schemes. This section further discusses the example of
the universal family of quartics in $\PP^3$, after summarizing some of its
properties. We work over a field $k$, but omit it in most notation\footnotemark{}, e.g.\ we write $\PP^3$ for $\PP^3_k$.

\footnotetext{Most instances of the letter $k$ will be used to denote a natural number instead.}

Denote by $\schemeofquartics$ the complete linear system
$\PP(H^0(\PP^3, \mc{O}(4)))$
of quartics in $\PP^3$. The quartics $\mf{X}_t \subseteq \PP^3$ parametrized by the $t\in\schemeofquartics$ form a universal family
$\pi \from \mf{X} \to \schemeofquartics$ with fibers $\mf{X}_t$.
The family $\mf{X}$ is a closed subscheme of
$\PP^3 \times \schemeofquartics$. The morphism $\pi$ is projective and flat. 

% which can be seen as follows. Let the index $I$ range over the tuples of the 
% form $(i_0,i_1,i_2,i_3)$ with $i_j \geq 0$ and $\sum i_j = 4$, and let $x_I$ 
% denote the $I$-th projective coordinate of $\schemeofquartics$. For $j=0,\dotsc
% ,3$, let $x_j$ denote the $j$-th coordinate of $\PP^3$. Then the family $\mf{X}
% $ is cut out by the section $\sum_{I} x_{I} x^{I}$ of the line bundle $\mc{O}(4
% ) \boxtimes \mc{O}(1)$ on $\PP^3 \times \schemeofquartics$.

Throughout, the homogeneous coordinates of $\PP^3$ will be denoted by
$x_i$, $i=0,\dotsc,4$.

We define the line bundle $\mc{L}$ on $\mf{X}$ as the restriction of
$\mc{O}(1)\boxtimes\mc{O}$ to $\mf{X}$. In other words\footnotemark{}, the bundle $\mc{L}$ is the pullback of
$\mc{O}(1)$ under the canonical projection $\mf{X} \to \PP^3$.

\footnotetext{For a fiber product $X \xleftarrow{p} X\times Y\xrightarrow{q} Y$ and sheaves $\mc{F}$ and $\mc{G}$ on $X$ resp.\ $Y$, write $\mc{F}\boxtimes \mc{G}\coloneqq \pullb{p} \mc{F} \otimes \pullb{q} \mc{G}$.}

\begin{proposition} \label{quartics-base-change}
	Let $k\geq 1$. The following statements hold:

	\begin{enumerate}
	\item If $q\in \schemeofquartics$ then
	$h^0 ( \mf{X}_q, \mc{L}^{\otimes k}|_q) = \binom{k+3}{3} - \binom{k-1}{3}$.
	In particular this number is independent of the point $q$.

	\item The sheaf
	$\pushf{\pi}\mc{L}^{\otimes k}$
	is locally free of rank
	$\binom{k+3}{3} - \binom{k-1}{3}$.

	\item For all cartesian diagrams of the form 
	\[
	\cartesiansquare{\mf{X}_{Z}}{}{\mf{X}}{\pi_Z}{\pi}{Z}{\rho}{\schemeofquartics}
	\]
	we have
	$\pullb{\rho}\pushf{\pi}\mc{L}^{\otimes k}
	\simeq
	\pushf{(\pi_Z)}\mc{L}^{\otimes k}_Z$.
	\end{enumerate}
\end{proposition}

\begin{proof}
	For the first statement, see the proof of
	\cite[Proposition 4.1]{hemminghaus-verlinde-bundles}. The others follow from Grauert's Theorem
	\cite[{}28.1.5]{vakil-algebraic-geometry}.
	% Todo: replace with own short proof
	% of the sections of the sheaf $\mc{L}$ at the fiber of a point $q\in \schemeofquartics$ does not depend on $q$. Indeed, the fiber $\mc{X}_q$ is a hypersurface of degree $4$ embedded in the projective space $\PP^{3}_{\kappa(q)}$. Its structure sequence on $\PP^{3}_{\kappa(q)}$ is
	% \[\ses{\mc{O}(-d)}{\mc{O}}{\mc{O}_{X_q}}.\]
	% Twisting with $\mc{O}(k)$ yields
	% \[\ses{\mc{O}(k-d)}{}{}\]
\end{proof}

Let $T \subseteq \schemeofquartics$ be the closed subscheme defined as the image of a linear embedding
$\PP^1_K\to \schemeofquartics$, with $K$ an extension field of $k$.
We call $T$ a \emph{pencil} of quartics. Its universal family is the scheme
$\mf{X}_{\PP^1_K}$, which comes with the polarization
$\mc{L}_{\PP^1_K}$. The situation is summarized in the picture below:
\[
\cartesiansquare{\mf{X}_{\PP^1_K}}{}{\mf{X}}
				  {}  {\pi}
				  {\PP^1_K}{}{\schemeofquartics}
\]	
For $k\geq 1$, we define the $k$-th Verlinde bundles
$V_k \coloneqq \pushf{\pi}\mc{L}^{\otimes k}$
and
$V_{k,T} \coloneqq \pushf{(\pi_{\PP^1})}{\mc{L}_{\PP^1}^{\otimes k}}$.
These bundles are related by
$V_{k}|_T = V_{k,T}$
using \Cref{quartics-base-change}.

\begin{proposition} \label{verlinde-exact-sequence}
	There exists a short exact sequence of coherent
	$\mc{O}_{\schemeofquartics}$-modules 
	\[
	\ses{\mc{O}(-1) \otimes H^0(\PP^3, \mc{O}(k-4))}
	    {\mc{O} \otimes H^0(\PP^3, \mc{O}(k))}
	    {V_k}.
	\]
	Let $I_d$ range over the tuples of the form
	$(i_0,\dotsc,i_3)$
	with
	$\sum i_j = d$.
	The first map is then given by 
	$\xi \otimes x^{I_{k-4}}
	\mapsto
	\sum_{I_4} \xi x^{I_4} \otimes x^{I_{k-4}+I_4}$.   
	\end{proposition}

\begin{proof}
	See \cite[Proposition 4.2]{hemminghaus-verlinde-bundles}.
	% Ergänz' mit own proof.
\end{proof}

\begin{remark}
	Let $T$ be a pencil of quartics.

	\begin{enumerate}
	\item The sequence from \Cref{verlinde-exact-sequence} restricts to a sequence
	\[
	\ses{\mc{O}(-1) \otimes H^0(\PP^3, \mc{O}(k-4))}
	    {\mc{O} \otimes H^0(\PP^3, \mc{O}(k))}
	    {V_{k,T}}
	\]
	over $\PP^1$.

	\item The vector bundle $V_{k,T}$ has determinant $\mc{O}(\binom{k-1}{3})$ and rank $\binom{k+3}{3} - \binom{k-1}{3}$.
	\end{enumerate}
\end{remark}

\begin{definition}
Let $k\geq 1$.

\begin{enumerate}
\item A \emph{type candidate} for $V_k$ is a non-increasing tuple $(d_1,\dotsc,d_{r^{(k)}})$ of non-negative integers with
$r^{(k)}=\binom{k+3}{3}-\binom{k-1}{3}$ and $\sum d_i = \binom{k-1}{3}$.

\item The \emph{general type candidate} for $V_k$ is the unique type candidate for $V_k$ of the form
$(b^{(k)} + 1,\dotsc,b^{(k)} + 1,b^{(k)},\dotsc,b^{(k)})$. The integer $b^{(k)}$ is determined by the equation
$\binom{k-1}{3} = b^{(k)} r^{(k)} + a$, with $a < r^{(k)}$ becoming the number of occurences of $b^{(k)} +1$. 

\item Let $T$ be a pencil of quartics. The \emph{type} of $V_{k,T}$ is the unique type candidate $(d_i)$ such that
$V_{k,T} \simeq \bigoplus_i \mc{O}(d_i)$.

\item We say that $V_{k,T}$ has \emph{general type} if its type is a general type candidate.

\end{enumerate}
\end{definition}

% \item A \emph{type candidate} for $V_k$ is a tuple \footnotemark{}
% $(d_1,\dotsc,d_{r(k)})$
% of non-negative integers
% with $r(k)=\binom{k+3}{3}-\binom{k-1}{3}$ and
% $\sum d_i = \binom{k-1}{3}$.

% \footnotetext{i.e.\ an equivalence class of tuples up to reordering of the entries}

% \item Of the type candidates for $V_k$, there exists a unique one of the form
% $(d_k,\dotsc,d_k,d_{k+1},\dotsc,d_{k+1})$, namely with $d_k=$
% The \emph{general type candidate} for $V_k$ is the unique\footnotemark{} type candidate for $V_k$ of the form
% $(d,\dotsc,d,d+1,\dotsc, d+1)$.
% %maybe mention relationship with specialization on ext^1.
% \footnotetext{For fixed $r,s>0$, the equation $dr + b = s$ has only one non-negative solution $(d,b)$ with $b<r$.}


The points of $\Gr(2,35)$ correspond to the pencils of quartics $T \subseteq \schemeofquartics$ in the following way. Let $P$ the universal $\PP^1$-bundle over $\Gr(2,35)$. It comes equipped with a projection map $P \to \schemeofquartics$ such that for all pencils of quartics $T$ there exists a unique point $t\in \Gr(2,35)$
such that the image of the fiber $P_t$ in $\schemeofquartics$ is $T$.
\[
\begin{tikzcd} [ampersand replacement = \&]
P_t \arrow{r} \arrow{d} \MySymb{\times}{dr} \& P \arrow{r}{p} \arrow{d}{\phi}\&
\schemeofquartics \\
\Spec(\kappa(t))\arrow{r} \& \Gr(2,35) \& \ 
\end{tikzcd}
\]
For $t\in\Gr(2,35)$ corresponding to the pencil $T$, we write $V_{k,t}\coloneqq V_{k,T}$.

\begin{definition}
Let $k \geq 1$ and $(d_i)$ be a type candidate for $V_k$. We define the set
$Z_{(d_i)}$ of all points $t\in\Gr(2,35)$ such that $V_{k,t}$ has type $(d_i)$. For the set of points $t$ where $V_{k,t}$ has general type, we also write $Z_{\text{gen}}$.
\end{definition}

% \begin{proposition}	
% The set $Z_{\text{gen}}$, and its complement, are not empty.
% \end{proposition}

% \begin{proposition}
% The set $Z_{\text{gen}}$ is Zariski-open. Its complement is a determinantal variety of codimension at least ()().
% \end{proposition}

% \begin{proof}
% After dualizing and pulling back the exact sequence from
% \Cref{verlinde-exact-sequence},



% The map $\phi$ is flat and proper, the scheme $\Gr(2,35)$ is reduced and locally Noetherian.
% \end{proof}

\begin{proposition}
	The set $Z_{\text{gen}}$ is Zariski open. Its complement is the union
	\[
		\Supp(R^1 \pushf{\phi}\pullb{p} V_{k}(-b^{(k)}-1)) \cup
		\Supp(R^1 \pushf{\phi}(\pullb{p} V_{k}(-b^{(k)})\dual)).
	\]
\end{proposition}
\begin{proof}
	We begin by characterizing the set $\generallocus$ via cohomology. Let $t\in \Gr(2,35)$, write $V_{k,t}=\bigoplus_{i=1}^r \mc{O}(d_i)$ and $b\coloneqq b^{(k)}$. We have $t\in \generallocus$ if and only if $b\leq d_i \leq b+1$ for all $i$, which holds if and only if
	$
	H^1(P_t, V_{k,t}(-b-1))
	=
	H^1(P_t, V_{k,t}(-b)\dual)
	=0.
	$

	Next, we want to apply the Cohomology and Base Change Theorem \cite[{}28.1.6]{vakil-algebraic-geometry} to the map 
	$\phi \from P \to \Gr(2,25)$, which is a $\PP^1$-bundle, proper and flat. The last property ensures that locally free sheaves on $P$ are flat over $\Gr(2,35)$.

	For all $t\in \Gr(2,35)$ we have
	$
	h^2(P_t, \pullb{p}V_{k,t}(-b-1)) = 0
	\text{ and }
	h^2(P_t,\pullb{p}V_{k,t}(-b)\dual)=0.
	$
	Since the sheaves $\pullb{p}V_{k,t}(-b-1)$ and $\pullb{p}V_{k,t}(-b)\dual$ are locally free and coherent, we have 
	\[(R^1 \pushf{\phi}\pullb{p} V_{k}(-b-1))_t = H^1(P_t,V_{k,t}(-b-1))\]
	\text{ and }
	\[
	(R^1 \pushf{\phi}(\pullb{p} V_{k}(-b)\dual))_t = H^1(P_t,V_{k,t}(-b)\dual).
	\]

	By the previous characterization, we have
	\[
		\Gr(2,35) \setminus \generallocus =
		\Supp(R^1 \pushf{\phi}\pullb{p} V_{k}(-b-1)) \cup
		\Supp(R^1 \pushf{\phi}(\pullb{p} V_{k}(-b)\dual)),
	\]
	which is a Zariski closed set.
\end{proof}

\begin{proposition} \label{supp-are-det-varieties}
The sets
$\Supp(R^1 \pushf{\phi}\pullb{p} V_{k}(-b^{(k)}-1))$ and
$\Supp(R^1 \pushf{\phi}(\pullb{p} V_{k}(-b^{(k)})\dual))$ 
are determinantal varieties.
\end{proposition}
\begin{proof}
To simplify notation, set
$r_1 \coloneqq \dim H^0(\PP^3, \mc{O}(k)),
r_2 \coloneqq \dim H^0(\PP^3, \mc{O}(k-4))$ and $b\coloneqq b^{(k)}$,
and rewrite the exact sequence from \Cref{verlinde-exact-sequence} as
\begin{align} \label{verlinde-simplified-exact-sequence}
\ses{\mc{O}(-1)^{r_2}}{\mc{O}^{r_1}}{V_k}. \tag{$\star$}
\end{align}
Twisting the sequence (\ref{verlinde-simplified-exact-sequence}) with $\mc{O}(-b-1)$ and pulling back to $P$ gives an exact sequence
\[
0
\to  {\pullb{p}\mc{O}(-b-2)^{r_2}}
\to  {\pullb{p}\mc{O}(-b-1)^{r_1}}
\to  {\pullb{p}V_k(-b-1)}
\to  0.
\]
For all $t\in \Gr(2,35)$ we have
$h^2(P_t, \mc{O}(-b-2)^{r_2}) = 0$,
hence
$R^2\pushf{\phi}\pullb{p}\mc{O}(-b-2)^{r_2} = 0$
and applying $\pushf{\phi}$ to the above sequence gives an exact sequence
\[
R^1\pushf{\phi}\pullb{p}\mc{O}(-b-2)^{r_2}
\xto{\alpha}
R^1\pushf{\phi}\pullb{p}\mc{O}(-b-1)^{r_1} 
\to
R^1\pushf{\phi}\pullb{p} V_k(-b-1)
\to 0.
\]
Note that since the numbers
$
h^{1}_{2}\coloneqq h^1(P_t, \mc{O}(-b-2)^{r_2})
\text{ and }
h_{1}^{1}\coloneqq h^1(P_t, \mc{O}(-b-1)^{r_1})
$
do not depend on the point $t$, Grauert's Theorem applies, and the first two terms of the above sequence are locally free and coherent of rank $h_1^2$ and $h_1^1$, respectively. Since taking the fiber is right-exact, we see that for all $t$ we have
$(R^1\pushf{\phi}\pullb{p} V_k(-b-1))_t \neq 0$ if and only if $\coker(\alpha_t) \neq 0$. Concluding, we have
\[
\Supp(R^1\pushf{\phi}(\pullb{p} V_k(-b-1)))
= \{t : \rank (\alpha_t)\leq h^{1}_1 - 1\}.
\]
As a final remark, note that $h^1_1 = b r_1 = b \binom{k+3}{3}.$

The proof for the second assertion is similar. We start with the sequence (\ref{verlinde-simplified-exact-sequence}), twist with $\mc{O}(-b)$, take duals, pull back to $P$, and apply $\pushf{\phi}$. Since for all $t\in \Gr(2,35)$ we have $h^1(P_t, \mc{O}(b)^{r_1})=0$, we obtain an exact sequence
\[
	{\pushf{\phi}\pullb{p}\mc{O}(b)^{r_1} }
\xto{\beta}	{\pushf{\phi}\pullb{p}\mc{O}(b+1)^{r_2}}
\to	{R^1 \pushf{\phi}(\pullb{p} V_{k}(-b)\dual)}
\to 0.
\]
Since the numbers
$
h^0_1 \coloneqq h^0(P_t,\mc{O}(b)^{r_1}) \text{ and }
h^0_2 \coloneqq h^0(P_t, \mc{O}(b+1)^{r_2})
$
do not depend on the point $t$, again by Grauert's Theorem the first two terms of the sequence are locally free of rank $h^0_1$ and $h^0_2$, respectively. As before, we obtain the characterization
\[
	\Supp(R^1 \pushf{\phi}(\pullb{p} V_{k}(-b)\dual))
	= \{t : \rank (\beta_t)\leq h^{0}_2 - 1\}.
\]
Here, we have $h_2^0 = (b+2)r_2 = (b+2)\binom{k-1}{3}.$
% Gilt Serre-dualität auch für Familien?
\end{proof}

\begin{definition}
	For type candidates $(d_i)$ and $(d'_i)$ we define the expression $(d'_i) \geq (d_i)$ to mean
	\[
		\sum_{i=1}^s d'_i \geq \sum_{i=1}^s d_i \text{ for all $s=1,\dotsc, r^{(k)}$}.
	\]
%%TODO	[-> Harder-Narasimhan-Polynome]
\end{definition}

\begin{proposition}
	Let $(d_i)$ be a type candidate for $V_k$. The set $\widehat Z_{(d_i)} \coloneqq \bigcup_{(d'_i)\geq(d_i)} Z_{(d'_i)}$ is the intersection of (at most) $r^{(k)}$ determinantal varieties.
\end{proposition}

\begin{proof}
	For every type candidate $(d'_i)$, the sum $\sum_{i=1}^s d'_i$ is the largest sum of $s$ entries of $(d'_i)$.  Hence we have 
	\[
		\widehat Z_{(d_i)} = \bigcap_{s=1}^{r^{(k)}} \{t : h^0((\textstyle{\bigwedge}^{s} V_{t,k})(-\textstyle{\sum}^s d_i)) > 0\}.
	\]
	With Serre duality and the Cohomology and Base Change theorem we write the sets in the intersection as 
	\[
		\Supp(R^1 \pushf \phi (\pullb p (\textstyle{\bigwedge}^{s} V_k\dual )(\textstyle{\sum}^s d_i - 2))),
	\]
	which is a determinantal variety by an argument similar to the second part of the proof of \Cref{supp-are-det-varieties}. One just has to note that $h^1(\PP^1,\mc{O}(\sum^s d_i - 2)) = 0$ for all $s$.
\end{proof}

\begin{proposition}
Of the five type candidates
\[
(1,1,1,1,0,\dotsc,0),\ (2,1,1,0,\dotsc,0),\ (2,2,0,\dotsc,0),\ (3,1,0,\dotsc,0),\ (4,0,\dotsc,0)
\]
for $V_5$, only the first two occur as types of some $V_{5,t}$.
\end{proposition}

\begin{proof}
% \newcommand{\mco}{\multicolumn{1}{c}}
% A $K$-point $q$ of $\Gr(2,35)$ is given by a matrix of the form
% \[
% 	\begin{pmatrix}
% 		\lambda_0 & \cdots & \lambda_{34} \\
% 		\mu_0 & \cdots & \mu_{34}
% 	\end{pmatrix},
% \]
% with $\lambda_i,\mu_i \in K,$ up to elementary row operations. Restricting the sequence from \Cref{verlinde-exact-sequence} to the quartic defined by $q$, we see that the bundle $V_{5,q}$ is the cokernel of the matrix $A \in \Mat(56\times 4, \Gamma(\PP^1, \mc{O}(1))$, given as follows. Let $s,t$ denote the homogeneous coordinates of $\PP^1$ and let $I_d$ range over the tuples of the form $(i_0,\dotsc,i_3)$ with $\sum i_j = d.$ The $(I_5,j)$-th entry of $A$ is
% $s\lambda_{I_4} + t\mu_{I_4}$ if $x_jx^{I_4} = x^{I_5}$ and $0$ else.

% Because of the invariance under elementary row operation, the point $q$ can be defined by a matrix of the form
% \[
% 	\left(
% 		\begin{array}{c|cccccccc}
% 			\multirow{2}{*}{0} & 1 & \lambda_1 & \cdots & \lambda_k & 0 & \lambda_{k+1} & \cdots & \lambda_{k+l} \\
% 			& 0 & 0 & \cdots & 0 & 1 & \mu_1 & \cdots & \mu_l
% 		\end{array}
% 	\right).
% \]
% We hence start with the matrix $A$ having the form
% \[
% 	\left(
% 		\begin{array}{c|ccc}
% 			s & 0 & 0 & 0 \\
% 			\cline{1-4}
% 			\lambda_1 s & & & \\
% 			\vdots & & \ast & \\
% 			\lambda_k s & & & \\
% 			t & & & \\
% 			\cline{1-4}
% 			\multicolumn{1}{c}{} & \ast & &
% 		\end{array}
% 	\right),
% \]
% with $\ast$ for now not specified, but not arbitrary. 

% We may perform elementary row and column operations without changing $\coker(A).$ Using \Cref{rem:exact-sequence-exists}, the goal is to show that the matrix $A$ can be modified to take one of the following forms:

% \[
% 	\left(
% 		\begin{array}{cccc}
% 			s &   &   &   \\
% 			t & s &   &   \\
% 			  & t &   &   \\
		
% 			  &   & s &   \\
% 			  &   & t &   \\
			
% 			  &   &   & s \\
% 			  &   &   & t \\
% 			\hline
% 			 & \multicolumn{1}{c}{0} & \multicolumn{1}{c}{} & 
% 		\end{array}
% 	\right), \quad
% 	\left(
% 		\begin{array}{cccc}
% 			s &   &   &   \\
% 			t &   &   &   \\
% 			  & s &   &   \\
% 			  & t &   &   \\
% 			  &   & s &   \\
% 			  &   & t &   \\
% 			  &   &   & s \\
% 			  &   &   & t \\
% 			\hline
% 			 & \multicolumn{1}{c}{0} & \multicolumn{1}{c}{} & 
% 		\end{array}
% 	\right).
% \]

% We see that this is possible in \Cref{types-for-v-five}.
% \end{proof}

% \begin{computation} \label{types-for-v-five}
% 	[Vielleicht abstrakteres Argument hier]
% \end{computation}
This is a special case of \Cref{no-big-types-general}.
\end{proof}