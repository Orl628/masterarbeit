%!TEX root = ./master.tex
\section{Verlinde Bundles on Pencils of Quartics}

%\begin{definition} A \emph{line} is an embedding $l\colon \mathbb{P}^1 \to X$ into a projective scheme $X$ such that $l^*\mathcal{O}(1) = \mathcal{O}(1)$.
%\end{definition}

The thesis \cite{hemminghaus-verlinde-bundles} studies Verlinde bundles for
families of polarized schemes. This section further discusses the example of
the universal family of quartics in $\PP^3$, after summarizing some of its
properties. We work over a field $k$, but omit it in most notation\footnotemark{}, e.g.\ we write $\PP^3$ for $\PP^3_k$.

\footnotetext{Most instances of the letter $k$ will be used to denote a natural number instead.}

Denote by $\schemeofquartics$ the complete linear system
$\PP(H^0(\PP^3, \mc{O}(4)))$
of quartics in $\PP^3$. The quartics $\mf{X}_t \subseteq \PP^3$ parametrized by the $t\in\schemeofquartics$ form a universal family
$\pi \from \mf{X} \to \schemeofquartics$ with fibers $\mf{X}_t$.
The family $\mf{X}$ is a closed subscheme of
$\PP^3 \times \schemeofquartics$. The morphism $\pi$ is projective and flat. 

% which can be seen as follows. Let the index $I$ range over the tuples of the 
% form $(i_0,i_1,i_2,i_3)$ with $i_j \geq 0$ and $\sum i_j = 4$, and let $x_I$ 
% denote the $I$-th projective coordinate of $\schemeofquartics$. For $j=0,\dotsc
% ,3$, let $x_j$ denote the $j$-th coordinate of $\PP^3$. Then the family $\mf{X}
% $ is cut out by the section $\sum_{I} x_{I} x^{I}$ of the line bundle $\mc{O}(4
% ) \boxtimes \mc{O}(1)$ on $\PP^3 \times \schemeofquartics$.

Throughout, the homogeneous coordinates of $\PP^3$ will be denoted by
$x_i$, $i=0,\dotsc,4$.

We define the line bundle $\mc{L}$ on $\mf{X}$ as the restriction of
$\mc{O}(1)\boxtimes\mc{O}$ to $\mf{X}$. In other words\footnotemark{}, the bundle $\mc{L}$ is the pullback of
$\mc{O}(1)$ under the canonical projection $\mf{X} \to \PP^3$.

\footnotetext{For a fiber product $X \xleftarrow{p} X\times Y\xrightarrow{q} Y$ and sheaves $\mc{F}$ and $\mc{G}$ on $X$ resp.\ $Y$, write $\mc{F}\boxtimes \mc{G}\coloneqq \pullb{p} \mc{F} \otimes \pullb{q} \mc{G}$.}

\begin{proposition} \label{quartics-base-change}
	Let $k\geq 1$. The following statements hold:

	\begin{enumerate}
	\huyitem If $q\in \schemeofquartics$ then
	$h^0 ( \mf{X}_q, \mc{L}^{\otimes k}|_q) = \binom{k+3}{3} - \binom{k-1}{3}$.
	In particular this number is independent of the point $q$.

	\huyitem The sheaf
	$\pushf{\pi}\mc{L}^{\otimes k}$
	is locally free of rank
	$\binom{k+3}{3} - \binom{k-1}{3}$.

	\huyitem For all cartesian diagrams of the form 
	\[
	\cartesiansquare{\mf{X}_{Z}}{}{\mf{X}}{\pi_Z}{\pi}{Z}{\rho}{\schemeofquartics}
	\]
	we have
	$\pullb{\rho}\pushf{\pi}\mc{L}^{\otimes k}
	\simeq
	\pushf{(\pi_Z)}\mc{L}^{\otimes k}_Z$.
	\end{enumerate}
\end{proposition}

\begin{proposition} \label{verlinde-base-change}
	Let $k\geq 1$. The following statements hold:

	\begin{enumerate}
	\huyitem If $q\in \schemeofquartics$ then
	$h^0 ( \mf{X}_q, \mc{L}^{\otimes k}|_q) = \binom{k+3}{3} - \binom{k-1}{3}$.
	In particular this number is independent of the point $q$.

	\huyitem The sheaf
	$\pushf{\pi}\mc{L}^{\otimes k}$
	is locally free of rank
	$\binom{k+3}{3} - \binom{k-1}{3}$.

	\huyitem For all cartesian diagrams of the form 
	\[
	\cartesiansquare{\mf{X}_{Z}}{}{\mf{X}}{\pi_Z}{\pi}{Z}{\rho}{\schemeofquartics}
	\]
	we have
	$\pullb{\rho}\pushf{\pi}\mc{L}^{\otimes k}
	\simeq
	\pushf{(\pi_Z)}\mc{L}^{\otimes k}_Z$.
	\end{enumerate}
\end{proposition}

\begin{proof}
	The proof for the first statement is found in
	\cite[Proposition 4.1]{hemminghaus-verlinde-bundles}
	and reproduced below. The others follow from Grauert's Theorem
	\cite[{}28.1.5]{vakil-algebraic-geometry}.
	% Todo: replace with own short proof
	% of the sections of the sheaf $\mc{L}$ at the fiber of a point $q\in \schemeofquartics$ does not depend on $q$. Indeed, the fiber $\mc{X}_q$ is a hypersurface of degree $4$ embedded in the projective space $\PP^{3}_{\kappa(q)}$. Its structure sequence on $\PP^{3}_{\kappa(q)}$ is
	% \[\ses{\mc{O}(-d)}{\mc{O}}{\mc{O}_{X_q}}.\]
	% Twisting with $\mc{O}(k)$ yields
	% \[\ses{\mc{O}(k-d)}{}{}\]
\end{proof}	

Let $T \subseteq \schemeofquartics$ be the closed subscheme defined as the image of a linear embedding
$\PP^1_K\to \schemeofquartics$, with $K$ an extension field of $k$.
We call $T$ a \emph{pencil} of quartics. Its universal family is the scheme
$\mf{X}_{\PP^1_K}$, which comes with the polarization
$\mc{L}_{\PP^1_K}$. The situation is summarized in the picture below:
\[
\cartesiansquare{\mf{X}_{\PP^1_K}}{}{\mf{X}}
				  {}  {\pi}
				  {\PP^1_K}{}{\schemeofquartics}
\]	
\begin{definition}For $k\geq 1$, we define the $k$-th Verlinde bundles
$V_k \coloneqq \pushf{\pi}\mc{L}^{\otimes k}$
and
$V_{k,T} \coloneqq \pushf{(\pi_{\PP^1})}{\mc{L}_{\PP^1}^{\otimes k}}$.
These bundles are related by
$V_{k}|_T = V_{k,T}$
using \Cref{quartics-base-change}.
\end{definition}

\begin{proposition} \label{verlinde-exact-sequence}
	There exists a short exact sequence of coherent
	$\mc{O}_{\schemeofquartics}$-modules 
	\[
	\ses{\mc{O}(-1) \otimes H^0(\PP^3, \mc{O}(k-4))}
	    {\mc{O} \otimes H^0(\PP^3, \mc{O}(k))}
	    {V_k}.
	\]
	Let $I_d$ range over the tuples of the form
	$(i_0,\dotsc,i_3)$
	with
	$\sum i_j = d$.
	The first map is then given by 
	$\xi \otimes x^{I_{k-4}}
	\mapsto
	\sum_{I_4} \xi x^{I_4} \otimes x^{I_{k-4}+I_4}$.   
	\end{proposition}

\begin{proof}
	See \cite[Proposition 4.2]{hemminghaus-verlinde-bundles}.
	% Ergänz' mit own proof.
\end{proof}

\begin{remark}
	Let $T$ be a pencil of quartics.

	\begin{enumerate}
	\item The sequence from \Cref{verlinde-exact-sequence} restricts to a sequence
	\begin{equation} \label{verlinde-exact-sequence-p1}
	\ses{\mc{O}(-1) \otimes H^0(\PP^3, \mc{O}(k-4))}
	    {\mc{O} \otimes H^0(\PP^3, \mc{O}(k))}
	    {V_{k,T}}
	\end{equation}
	over $\PP^1$.

	\item The vector bundle $V_{k,T}$ has determinant $\mc{O}(\binom{k-1}{3})$ and rank $\binom{k+3}{3} - \binom{k-1}{3}$.

	\item Let $V_{k,T} \simeq \bigoplus_i \mc{O}(d_i)$ be a splitting of $V_{k,T}$ over $\PP^1$. By the sequence \cref{verlinde-exact-sequence-p1}, we have $d_i \geq 0$.
	\end{enumerate}
\end{remark}

\begin{definition}
Let $k\geq 1$.

\begin{enumerate}
\item A \emph{type candidate} for $V_k$ is a non-increasing tuple $(d_1,\dotsc,d_{r^{(k)}})$ of non-negative integers with
$r^{(k)}=\binom{k+3}{3}-\binom{k-1}{3}$ and $\sum d_i = \binom{k-1}{3}$.

\item The \emph{general type candidate} for $V_k$ is the unique type candidate for $V_k$ of the form
$(b^{(k)} + 1,\dotsc,b^{(k)} + 1,b^{(k)},\dotsc,b^{(k)})$. The integer $b^{(k)}$ is determined by the equation
$\binom{k-1}{3} = b^{(k)} r^{(k)} + a$, with $a < r^{(k)}$ becoming the number of occurences of $b^{(k)} +1$.

\item Let $E$ be a locally free sheaf on $\PP^1$. The \emph{type} of $E$ is the unique non-increasing tuple $(d_1,\dotsc,d_{r^{(k)}})$ such that $E \simeq \bigoplus_i \mc{O}(d_i)$.

\item We say that $V_{k,T}$ has \emph{general type} if its type is a general type candidate.
\end{enumerate}

\begin{definition}
Let $P$ denote the universal $\PP^1$-bundle of the Grassmannian of lines $\Gr(2,H^0(\mc{O}(4))) = \GGr(1,\schemeofquartics)$, let $\phi\from P\to \GGr(1,\schemeofquartics)$ be the universal map and $p\from P\to \schemeofquartics$ the canonical projection.
\[
\begin{tikzcd} [ampersand replacement = \&]
P \arrow{r}{p} \arrow{d}{\phi}\& \schemeofquartics \\
\GGr(1,\schemeofquartics) \& 
\end{tikzcd}
\]
The mapping $t\mapsto P_t$ gives a canonical bijection between the points of $\GGr(1,\schemeofquartics)$ and the pencils of quartics in $\schemeofquartics$. For such $t$, we write $V_{k,t} \coloneqq V_{k,p(P_t)}$.
\end{definition}

\end{definition}

% \item A \emph{type candidate} for $V_k$ is a tuple \footnotemark{}
% $(d_1,\dotsc,d_{r(k)})$
% of non-negative integers
% with $r(k)=\binom{k+3}{3}-\binom{k-1}{3}$ and
% $\sum d_i = \binom{k-1}{3}$.

% \footnotetext{i.e.\ an equivalence class of tuples up to reordering of the entries}

% \item Of the type candidates for $V_k$, there exists a unique one of the form
% $(d_k,\dotsc,d_k,d_{k+1},\dotsc,d_{k+1})$, namely with $d_k=$
% The \emph{general type candidate} for $V_k$ is the unique\footnotemark{} type candidate for $V_k$ of the form
% $(d,\dotsc,d,d+1,\dotsc, d+1)$.
% %maybe mention relationship with specialization on ext^1.
% \footnotetext{For fixed $r,s>0$, the equation $dr + b = s$ has only one non-negative solution $(d,b)$ with $b<r$.}


% The points of $\Gr(2,35)$ correspond to the pencils of quartics $T \subseteq \schemeofquartics$ in the following way. Let $P$ the universal $\PP^1$-bundle over $\Gr(2,35)$. It comes equipped with a projection map $P \to \schemeofquartics$ such that for all pencils of quartics $T$ there exists a unique point $t\in \Gr(2,35)$
% such that the image of the fiber $P_t$ in $\schemeofquartics$ is $T$.
% \[
% \begin{tikzcd} [ampersand replacement = \&]
% P_t \arrow{r} \arrow{d} \MySymb{\times}{dr} \& P \arrow{r}{p} \arrow{d}{\phi}\&
% \schemeofquartics \\
% \Spec(\kappa(t))\arrow{r} \& \Gr(2,35) \& \ 
% \end{tikzcd}
% \]
% For $t\in\Gr(2,35)$ corresponding to the pencil $T$, we write $V_{k,t}\coloneqq V_{k,T}$.

\begin{definition}
Let $k \geq 1$ and $(d_i)$ be a type candidate for $V_k$. We define the set
$Z_{(d_i)}$ of all points $t\in\GGr(1,\schemeofquartics)$ such that $V_{k,t}$ has type $(d_i)$. For the set of points $t$ where $V_{k,t}$ has general type, we also write $Z_{\text{gen}}$.
\end{definition}

% \begin{proposition}	
% The set $Z_{\text{gen}}$, and its complement, are not empty.
% \end{proposition}

% \begin{proposition}
% The set $Z_{\text{gen}}$ is Zariski-open. Its complement is a determinantal variety of codimension at least ()().
% \end{proposition}

% \begin{proof}
% After dualizing and pulling back the exact sequence from
% \Cref{verlinde-exact-sequence},



% The map $\phi$ is flat and proper, the scheme $\Gr(2,35)$ is reduced and locally Noetherian.
% \end{proof}

\begin{proposition}
	The set $Z_{\text{gen}}$ is Zariski open. Its complement is the union
	\[
		\Supp(R^1 \pushf{\phi}\pullb{p} V_{k}(-b^{(k)}-1)) \cup
		\Supp(R^1 \pushf{\phi}(\pullb{p} V_{k}(-b^{(k)})\dual)).
	\]
\end{proposition}
\begin{proof}
	We begin by characterizing the set $\generallocus$ via cohomology. Let $t\in \GGr(1,\schemeofquartics)$, write $V_{k,t}=\bigoplus_{i=1}^r \mc{O}(d_i)$ and $b\coloneqq b^{(k)}$. We have $t\in \generallocus$ if and only if $b\leq d_i \leq b+1$ for all $i$, which holds if and only if
	$
	H^1(P_t, V_{k,t}(-b-1))
	=
	H^1(P_t, V_{k,t}(-b)\dual)
	=0.
	$

	Next, we want to apply the Cohomology and Base Change Theorem \cite[{}28.1.6]{vakil-algebraic-geometry} to the map 
	$\phi \from P \to \GGr(1,\schemeofquartics)$, which is a $\PP^1$-bundle, proper and flat. The last property ensures that locally free sheaves on $P$ are flat over $\GGr(1,\schemeofquartics)$.

	For all $t\in \GGr(1,\schemeofquartics)$ we have
	$
	h^2(P_t, \pullb{p}V_{k,t}(-b-1)) = 0
	\text{ and }
	h^2(P_t,\pullb{p}V_{k,t}(-b)\dual)=0.
	$
	Since the sheaves $\pullb{p}V_{k,t}(-b-1)$ and $\pullb{p}V_{k,t}(-b)\dual$ are locally free and coherent, we have 
	\[(R^1 \pushf{\phi}\pullb{p} V_{k}(-b-1))_t = H^1(P_t,V_{k,t}(-b-1))\]
	\text{ and }
	\[
	(R^1 \pushf{\phi}(\pullb{p} V_{k}(-b)\dual))_t = H^1(P_t,V_{k,t}(-b)\dual).
	\]

	By the previous characterization, we have
	\[
		\GGr(1,\schemeofquartics) \setminus \generallocus =
		\Supp(R^1 \pushf{\phi}\pullb{p} V_{k}(-b-1)) \cup
		\Supp(R^1 \pushf{\phi}(\pullb{p} V_{k}(-b)\dual)),
	\]
	which is a Zariski closed set.
\end{proof}
\begin{proposition} \label{supp-are-det-varieties}
The sets
$\Supp(R^1 \pushf{\phi}\pullb{p} V_{k}(-b^{(k)}-1))$ and
$\Supp(R^1 \pushf{\phi}(\pullb{p} V_{k}(-b^{(k)})\dual))$ 
are determinantal varieties in the sense of \cite[Ch.~II, §4]{arbarello-geometry-algebraic-curves}
\end{proposition}
\begin{proof}
To simplify notation, set
$r_1 \coloneqq \dim H^0(\PP^3, \mc{O}(k)),
r_2 \coloneqq \dim H^0(\PP^3, \mc{O}(k-4))$ and $b\coloneqq b^{(k)}$,
and rewrite the exact sequence from \Cref{verlinde-exact-sequence} as
\begin{align} \label{verlinde-simplified-exact-sequence}
\ses{\mc{O}(-1)^{r_2}}{\mc{O}^{r_1}}{V_k}. %\tag{$\star$}
\end{align}
Twisting the sequence \cref{verlinde-simplified-exact-sequence} with $\mc{O}(-b-1)$ and pulling back to $P$ gives an exact sequence
\[
0
\to  {\pullb{p}\mc{O}(-b-2)^{r_2}}
\to  {\pullb{p}\mc{O}(-b-1)^{r_1}}
\to  {\pullb{p}V_k(-b-1)}
\to  0.
\]
For all $t\in \GGr(1,\schemeofquartics)$ we have
$h^2(P_t, \mc{O}(-b-2)^{r_2}) = 0$,
hence
$R^2\pushf{\phi}\pullb{p}\mc{O}(-b-2)^{r_2} = 0$
and applying $\pushf{\phi}$ to the above sequence gives an exact sequence
\[
R^1\pushf{\phi}\pullb{p}\mc{O}(-b-2)^{r_2}
\xto{\alpha}
R^1\pushf{\phi}\pullb{p}\mc{O}(-b-1)^{r_1} 
\to
R^1\pushf{\phi}\pullb{p} V_k(-b-1)
\to 0.
\]
Note that since the numbers
$
h^{1}_{2}\coloneqq h^1(P_t, \mc{O}(-b-2)^{r_2})
\text{ and }
h_{1}^{1}\coloneqq h^1(P_t, \mc{O}(-b-1)^{r_1})
$
do not depend on the point $t$, Grauert's Theorem applies, and the first two terms of the above sequence are locally free and coherent of rank $h_1^2$ and $h_1^1$, respectively. Since taking the fiber is right-exact, we see that for all $t$ we have
$(R^1\pushf{\phi}\pullb{p} V_k(-b-1))_t \neq 0$ if and only if $\coker(\alpha_t) \neq 0$. Concluding, we have
\[
\Supp(R^1\pushf{\phi}(\pullb{p} V_k(-b-1)))
= \{t : \rank (\alpha_t)\leq h^{1}_1 - 1\}.
\]
As a final remark, note that $h^1_1 = b r_1 = b \binom{k+3}{3}.$

The proof for the second assertion is similar. We start with the sequence \cref{verlinde-simplified-exact-sequence}, twist with $\mc{O}(-b)$, take duals, pull back to $P$, and apply $\pushf{\phi}$. Since for all $t\in \GGr(1,\schemeofquartics)$ we have $h^1(P_t, \mc{O}(b)^{r_1})=0$, we obtain an exact sequence
\[
	{\pushf{\phi}\pullb{p}\mc{O}(b)^{r_1} }
\xto{\beta}	{\pushf{\phi}\pullb{p}\mc{O}(b+1)^{r_2}}
\to	{R^1 \pushf{\phi}(\pullb{p} V_{k}(-b)\dual)}
\to 0.
\]
Since the numbers
$
h^0_1 \coloneqq h^0(P_t,\mc{O}(b)^{r_1}) \text{ and }
h^0_2 \coloneqq h^0(P_t, \mc{O}(b+1)^{r_2})
$
do not depend on the point $t$, again by Grauert's Theorem the first two terms of the sequence are locally free of rank $h^0_1$ and $h^0_2$, respectively. As before, we obtain the characterization
\[
	\Supp(R^1 \pushf{\phi}(\pullb{p} V_{k}(-b)\dual))
	= \{t : \rank (\beta_t)\leq h^{0}_2 - 1\}.
\]
Here, we have $h_2^0 = (b+2)r_2 = (b+2)\binom{k-1}{3}.$
% Gilt Serre-dualität auch für Familien?
\end{proof}

\begin{definition} \label{def-compare-types}
	For type candidates $(d_i)$ and $(d'_i)$ we define the expression $(d'_i) \geq (d_i)$ to mean
	\[
		\sum_{i=1}^s d'_i \geq \sum_{i=1}^s d_i \text{ for all $s=1,\dotsc, r^{(k)}$}.
	\]
%%TODO	[-> Harder-Narasimhan-Polynome]
\end{definition}

% \begin{proposition}
% 	Let $(d_i)$ be a type candidate for $V_k$. The set $\widehat Z_{(d_i)} \coloneqq \bigcup_{(d'_i)\geq(d_i)} Z_{(d'_i)}$ is the intersection of (at most) $r^{(k)}$ determinantal varieties. In particular, the set $Z_{(d_i)}$ is locally closed.
% \end{proposition}

% \begin{proof}
% Let $t\in \GGr(1,\schemeofquartics)$ and $V_{k,t} = \bigoplus_{i=1}^{r^{(k)}} \mc{O}(d'_i)$. We have
% \[
% \bigwedge^s V_{k,t} = \bigoplus_{I} \mc{O}(d_I'),
% \]
% where $I$ runs over the subsets of $\{1,\dotsc, r^{(k)}\}$ of size $s$ and $d'_I\coloneqq \sum_{i\in I} d'_i$.
% For every type candidate $(d'_i)$, the sum $\sum_{i=1}^s d'_i$ is the largest sum of $s$ entries of $(d'_i)$. Since $d'_i \geq 0$, the condition $\sum_{i=1}^s d'_i \geq \sum_{i=1}^s d_i$ is equivalent to the condition $h^0((\textstyle{\bigwedge}^{s} V_{t,k})(-\textstyle{\sum}^s d_i)) > 0$.
% \[
% 	\widehat Z_{(d_i)} = \bigcap_{s=1}^{r^{(k)}} \{t : h^0((\textstyle{\bigwedge}^{s} V_{t,k})(-\textstyle{\sum}^s d_i)) > 0\}.
% \]
% With Serre duality and the Cohomology and Base Change theorem we write the sets of the intersection as 
% \[
% 	\Supp(R^1 \pushf \phi (\pullb p (\textstyle{\bigwedge}^{s} V_k\dual )(\textstyle{\sum}^s d_i - 2))),
% \]
% which is a determinantal variety by an argument similar to the second part of the proof of \Cref{supp-are-det-varieties}: start with the sequence \cref{verlinde-simplified-exact-sequence},  One just has to note that $h^1(\PP^1,\mc{O}(\sum^s d_i - 2)) = 0$ for all $s$, since at least one $d_i$ is nonzero. This gives the exact sequence
% % \[
% % {\pushf{\phi}\pullb{p}\mc{O}(b)^{r_1} }
% % \xto{\beta}	{\pushf{\phi}\pullb{p}\mc{O}(b+1)^{r_2}}
% % \to	{R^1 \pushf{\phi}(\pullb{p} V_{k}(-b)\dual)}
% % \to 0.
% % \]
% \end{proof}

\begin{proposition}
	Let $(d_i)$ be a type candidate for $V_k$. The set $\widehat Z_{(d_i)} \coloneqq \bigcup_{(d'_i)\geq(d_i)} Z_{(d'_i)}$ is Zariski-closed. In particular, the set $Z_{(d_i)}$ is locally closed.
\end{proposition}

\begin{proof}
Let $t\in \GGr(1,\schemeofquartics)$ and $V_{k,t} = \bigoplus_{i=1}^{r^{(k)}} \mc{O}(d'_i)$. We have
\[
\bigwedge^s V_{k,t} = \bigoplus_{I} \mc{O}(d_I'),
\]
where $I$ runs over the subsets of $\{1,\dotsc, r^{(k)}\}$ of size $s$ and $d'_I\coloneqq \sum_{i\in I} d'_i$.
For every type candidate $(d'_i)$, the sum $\sum_{i=1}^s d'_i$ is the largest sum of $s$ entries of $(d'_i)$. Since $d'_i \geq 0$, the condition $\sum_{i=1}^s d'_i \geq \sum_{i=1}^s d_i$ is equivalent to the condition $h^0((\textstyle{\bigwedge}^{s} V_{t,k})(-\textstyle{\sum}^s d_i)) > 0$.
\[
	\widehat Z_{(d_i)} = \bigcap_{s=1}^{r^{(k)}} \{t : h^0((\textstyle{\bigwedge}^{s} V_{t,k})(-\textstyle{\sum}^s d_i)) > 0\}.
\]
With Serre duality and the Cohomology and Base Change theorem we write the sets of the intersection as 
\[
	\Supp(R^1 \pushf \phi (\pullb p (\textstyle{\bigwedge}^{s} V_k\dual )(\textstyle{\sum}^s d_i - 2))),
\]
which is Zariski-closed.
% \[
% {\pushf{\phi}\pullb{p}\mc{O}(b)^{r_1} }
% \xto{\beta}	{\pushf{\phi}\pullb{p}\mc{O}(b+1)^{r_2}}
% \to	{R^1 \pushf{\phi}(\pullb{p} V_{k}(-b)\dual)}
% \to 0.
% \]
\end{proof}

\begin{corollary}
	Let $(b_i)$ and $(b'_i)$ be type candidates. If $Z_{(b_i)} \subseteq Z_{(b_i')}$ then $(b_i) \geq (b'_i)$.
\end{corollary}

\begin{proposition}
Of the five type candidates
\[
(1,1,1,1,0,\dotsc,0),\ (2,1,1,0,\dotsc,0),\ (2,2,0,\dotsc,0),\ (3,1,0,\dotsc,0),\ (4,0,\dotsc,0)
\]
for $V_5$, only the first two occur as types of some $V_{5,t}$.
\end{proposition}

\begin{proof}
% \newcommand{\mco}{\multicolumn{1}{c}}
% A $K$-point $q$ of $\Gr(2,35)$ is given by a matrix of the form
% \[
% 	\begin{pmatrix}
% 		\lambda_0 & \cdots & \lambda_{34} \\
% 		\mu_0 & \cdots & \mu_{34}
% 	\end{pmatrix},
% \]
% with $\lambda_i,\mu_i \in K,$ up to elementary row operations. Restricting the sequence from \Cref{verlinde-exact-sequence} to the quartic defined by $q$, we see that the bundle $V_{5,q}$ is the cokernel of the matrix $A \in \Mat(56\times 4, \Gamma(\PP^1, \mc{O}(1))$, given as follows. Let $s,t$ denote the homogeneous coordinates of $\PP^1$ and let $I_d$ range over the tuples of the form $(i_0,\dotsc,i_3)$ with $\sum i_j = d.$ The $(I_5,j)$-th entry of $A$ is
% $s\lambda_{I_4} + t\mu_{I_4}$ if $x_jx^{I_4} = x^{I_5}$ and $0$ else.

% Because of the invariance under elementary row operation, the point $q$ can be defined by a matrix of the form
% \[
% 	\left(
% 		\begin{array}{c|cccccccc}
% 			\multirow{2}{*}{0} & 1 & \lambda_1 & \cdots & \lambda_k & 0 & \lambda_{k+1} & \cdots & \lambda_{k+l} \\
% 			& 0 & 0 & \cdots & 0 & 1 & \mu_1 & \cdots & \mu_l
% 		\end{array}
% 	\right).
% \]
% We hence start with the matrix $A$ having the form
% \[
% 	\left(
% 		\begin{array}{c|ccc}
% 			s & 0 & 0 & 0 \\
% 			\cline{1-4}
% 			\lambda_1 s & & & \\
% 			\vdots & & \ast & \\
% 			\lambda_k s & & & \\
% 			t & & & \\
% 			\cline{1-4}
% 			\multicolumn{1}{c}{} & \ast & &
% 		\end{array}
% 	\right),
% \]
% with $\ast$ for now not specified, but not arbitrary. 

% We may perform elementary row and column operations without changing $\coker(A).$ Using \Cref{rem:exact-sequence-exists}, the goal is to show that the matrix $A$ can be modified to take one of the following forms:

% \[
% 	\left(
% 		\begin{array}{cccc}
% 			s &   &   &   \\
% 			t & s &   &   \\
% 			  & t &   &   \\
		
% 			  &   & s &   \\
% 			  &   & t &   \\
			
% 			  &   &   & s \\
% 			  &   &   & t \\
% 			\hline
% 			 & \multicolumn{1}{c}{0} & \multicolumn{1}{c}{} & 
% 		\end{array}
% 	\right), \quad
% 	\left(
% 		\begin{array}{cccc}
% 			s &   &   &   \\
% 			t &   &   &   \\
% 			  & s &   &   \\
% 			  & t &   &   \\
% 			  &   & s &   \\
% 			  &   & t &   \\
% 			  &   &   & s \\
% 			  &   &   & t \\
% 			\hline
% 			 & \multicolumn{1}{c}{0} & \multicolumn{1}{c}{} & 
% 		\end{array}
% 	\right).
% \]

% We see that this is possible in \Cref{types-for-v-five}.
% \end{proof}

% \begin{computation} \label{types-for-v-five}
% 	[Vielleicht abstrakteres Argument hier]
% \end{computation}
This is a special case of \Cref{no-big-types-general}.
\end{proof}

\subsection{Calculations in the Chow Ring}

To perform calculations in the Chow ring $A$ of $\GGr(1,\schemeofquartics)$, we follow the conventions found in \cite{eisenbud-harris-intersection-theory}. We assume $\characteristic(k) = 0$ for simplicity. The ring $A$ is generated by the Schubert classes $\sigma_{a,b}$ of the Schubert cycles
\[
	\Sigma_{a,b}(\mc{H})\coloneqq \{T \in \GGr(1,\schemeofquartics) : T \cap H \neq \leer, T \subseteq H'\},
\]
where $\mc{H} = (H\subset H')$ is a flag of linear subspaces of dimension $33-a$ resp.\ $34-b$ in the $34$-dimensional projective space $\schemeofquartics$. The class $\Sigma_{a,b}$ has codimension $a+b$, and we use the convention $\sigma_{a}\coloneqq \sigma_{a,0}$.

% By Kleiman's theorem, finding the product of $\sigma_{a,b}$ and some class $[Z]$ of complementary codimension amounts to calculating the cardinality $\alpha$ of the set $Z\cap \Sigma_{a,b}(\mc{H})$, where $\mc{H}$ is a general flag. Then $\sigma_{a,b} \cdot [Z] = \alpha \sigma_{33,33}$.

\begin{proposition}
	Let $Z \coloneqq Z_{(2,1,1,0,\dotsc)}$ be locus of points $t\in \GGr(1,\schemeofquartics)$ such that $V_{5,t}$ is not generic. In the Chow ring $A$, we have
	\begin{equation}
		[Z] = \sum_{a=0}^{11} \left({\binom{24-a}{3}}{\binom{a+1}{3}}-{\binom{25-a}{3}}{\binom{a}{3}}\right) \sigma_{33-a,10+a}. \label{class-of-locus}
	\end{equation}
\end{proposition}

\begin{proof}
	Define the subvariety $Q \subset \schemeofquartics$ as the image of the map $$f\from \schemeofsections{1} \times \schemeofsections{3} \to \schemeofquartics$$ defined by $f(g,h) = gh$. The map $f$ is birational on its image, since a general point of $Q$ has the form $gh$ with $h$ irreducible. By \Cref{nongeneral-type-shared-sections}, the variety $Z$ is the image of the finite map
	$$\phi \from \GGr(1,\schemeofsections{1}) \times \schemeofsections{3} \to \GGr(1,\schemeofquartics)$$
	defined by $\phi((sg_1 + tg_2)_{(s:t)},h) = (sg_1h+tg_2h)_{(s:t)}$. In particular, a line $T\subset \schemeofquartics$ belonging to $Z$ lies in $Q$.

	Since $f$ and $\phi$ are finite, we have $\dim(Q) = 22$ and $\dim(Z)=23$, while $\codim(Z)=43$. The Chow group $A^{43}$ is generated by the classes $\sigma_{33,10},\sigma_{32,11},\dotsc,\sigma_{22,21}$, while the complementary group $A^{23}$ is generated by
	$\sigma_{23,0},\sigma_{22,1},\dotsc,\sigma_{12,11}$. Write
	\[
		[Z] = \sum_{a=0}^{11} \alpha_{33-a,10+a} \sigma_{33-a,10+a}. 
	\]
	We have $\sigma_{33-a,10+a} \sigma_{23-a',a'} = 1$ if $a=a'$ and $0$ else. Hence, multiplying the above equation with $\sigma_{33-a',10+a'}$ and taking degrees gives
	$
	\alpha_{33-a,10+a} = \deg([Z]\cdot \sigma_{23-a,a}).
	$

	Using Giambelli's formula 
	$\sigma_{23-a,a}=\sigma_{23-a}\sigma_{a} - \sigma_{24-a}\sigma_{a-1}$ \cite[Prop.\ 4.16]{eisenbud-harris-intersection-theory}, we reduce to computing $\deg([Z] \cdot \sigma_{23-a}\sigma_{a})$ for $0\leq a\leq 11$.
	By Kleiman transversality, we have 
	\[
	\deg([Z] \cdot \sigma_{23-a}\sigma_{a}) = \abs{\{T\in \GGr(1,\schemeofquartics) : T \cap H \neq \leer, T \cap H' \neq \leer\}},
	\] where $H$ and $H'$ are general linear subspaces of $\schemeofquartics$ of dimension $10+a$ and $33-a$, respectively.

	To a point $p = g_p h_p \in Q$ with $g_p \in \schemeofsections{1}$ and $h_p \in \schemeofsections{3}$, associate a closed reduced subscheme $\Lambda_p$ containing $p$ as follows. If $h_p$ is irreducible, let $\Lambda_p$ be the image of the linear embedding $\schemeofsections{1}\times \{h_p\} \to \schemeofquartics$ given by $g \mapsto g h_p$. If $h_p = g'_p h'_p$ with $h'_p \in \schemeofsections{2}$ irreducible, let $\Lambda_p$ be the union of the images of the linear embeddings $\schemeofsections{1} \times \{h_p\} \to \schemeofquartics$ and $\schemeofsections{1} \times \{g_p h'_p\} \to \schemeofquartics$. These two linear subspaces meet exactly at $p$. Similarly, if $p$ is the product of four linear forms, define the space $\Lambda_p$ as the union $\bigcup_h \im(\schemeofsections{1} \times \{h\}\to \schemeofquartics)$, where $h$ runs over the four cubics arising as products of the linear factors of $p$.

	By the definition of $Z$, all lines $T\in Z$ lie in $Q$ and if $T$ meets the point $p$, then $T\subseteq \Lambda_p$.
	For $H\subseteq\schemeofquartics$ a linear subspace of dimension $10+a$, define $Q'\coloneqq H\cap Q$. For general $H$, the subscheme $Q'$ is a smooth subvariety of dimension $a-2$ such that a general point $p=gh$ of $Q'$ with $h\in \schemeofquartics$ has $h$ irreducible.
	For $a=0,1$, a general linear subspace $H$ does not intersect $Q$ at all, from which follows $\deg([Z]\cdot \sigma_{23}\sigma_{0}) =
	\deg([Z]\cdot \sigma_{22}\sigma_{1}) = 0$.

	% By the construction of the $\Lambda_p$, if a line $T\in Z$ meets the point $p$, then $T\subseteq \Lambda_p$. A general linear subspace $H$ of dimension $10+a$ intersects the variety $Q$ at a smooth subvariety $Q'$ of dimension $a-2$. Furthermore, it may be assumed that a general point $p=gh$ of $Q'$ with $h\in \schemeofsections{3}$ has $h$ irreducible. Indeed, the subvariety of $Q$ where $h$ is reducible has dimension $\dim(\schemeofsections{1} \times \schemeofsections{1} \times \schemeofsections{2}) = 15$, so the dimension of its intersection with a general $H$ is $a-9$.

	Next, we show that for general $H$, for each point $p\in Q'$ we have $\Lambda_p \cap H =\{p\}$. Let $\mc{H}$ denote the parameter space for $H$, \ie the Grassmannian $\GGr(10+a, 34)$. Define the closed subset $X\subseteq Q\times \mc{H}$ by
	$X\coloneqq \{(p,H):\dim(H\cap \Lambda_p)\geq 1\}$. The fibers of the induced map $X\to \mc{H}$ have dimension at least one. Hence, to prove that the desired condition on $H$ is an open condition, it suffices to prove $\dim(X) \leq \dim(\mc{H})$. The fiber of the map  $X\to Q$ over a point $p$ consists of the union of finitely many closed subsets of the form $X'_p = \{H\in \mc{H} : \dim(H\cap \Lambda'_p)\geq 1\}$, where $\Lambda'_p\isom \PP^3\subseteq \schemeofquartics$ is one of the components of $\Lambda_p$. The space $X'_p$ is a Schubert cycle
	\[
		\Sigma_{22-a,22-a} = \{H\in \Gr(11,35) : \dim(H \cap H_4) \geq 2\},
	\]
	with $H_4$ a four-dimensional subspace of $H^0(\mc{O}(4))$. The codimension of the cycle is $2(22-a)$, hence also $\codim(X_p) = 2(22-a)$. Finally, we have $\dim(\mc{H})-\dim(X) = \codim(X_p) - \dim(Q) = 22-2a \geq 0$ for $0\leq a \leq 11$.

	Next, let $\Lambda \coloneqq \bigcup_{p\in Q'} \Lambda_p = f(\schemeofsections{1}\otimes \pr_2 f^{-1}(Q'))$ and $\Lambda''\coloneqq \schemeofsections{1}\otimes \pr_2 f^{-1}(Q')$. By the choice of $H$, the map $f^{-1}(Q')\to Q'$ is birational and the map $f^{-1}(Q)\to \pr_2f^{-1}(Q)$ is even bijective. It follows that $\Lambda''$ and hence $\Lambda$ have dimension $a+1$. The intersection of $\Lambda$ with a general linear subspace $H'$ of dimension $33-a$ is a finite set of points. For each point $p\in Q'$, the linear subspace $H'$ intersects each component $\Lambda'_p$ of $\Lambda_p$ in at most one point. For each point $p'\in H'\cap\Lambda$ there exists a unique $p$ such that $p'\in\Lambda_p$. Furthermore, the only line $T\in Z$ meeting both $p$ and $H'$ is the one through $p$ and $p'$. If the intersection $H'\cap \Lambda_p$ is empty, then there will be no line meeting $p$ and $H'$. Hence, $\deg([Z]\cdot \sigma_{22-a}\sigma_{a})$ is the number of intersection points of $\Lambda$ with a general $H'$.

	Finally, the pre-image $f^{-1}(Q') = f^{-1}(H)$ is smooth for a general $H$ by Bertini's Theorem. If $\zeta$ is the class of a hyperplane section of $\schemeofquartics$ we have $f^*(\zeta) = \alpha + \beta$, , where $\alpha$ and $\beta$ are classes of hyperplane sections of $\schemeofsections{1}$ and $\schemeofsections{3}$, respectively. Since $\pr_2$ and $f$ have degree one we compute:
	\begin{align*}
	[\Lambda''] & = [\pr_2^{-1}\pr_2 f^{-1}(H)] \\
	& = \pr_2^*[\pr_2 f^{-1}(H)] \\
	& = \pr_2^* \pr_{2,*}[f^{-1}(H)] \\
	& = \pr_2^*\pr_{2,*}f^*[H] \\
	& = \pr_2^* \pr_{2,*}(\alpha + \beta)^{24-a} \\
	& = \textstyle{\binom{24-a}{3}}\pr_2^* \beta^{21-a} \\
	& = \textstyle{\binom{24-a}{3}}\beta^{21-a}
	\end{align*}

	Hence, by the push-pull formula:
	\[
		\deg([\Lambda]\cdot H') = \deg([\Lambda'']\cdot(\alpha+\beta)^{a+1}) = \binom{24-a}{3}\binom{a+1}{3}
	\]
	We then use Giambelli's formula to obtain Equation \cref{class-of-locus}.

	% To compute this number, we compute the coefficient $\lambda$ of the term of highest degree of the class
	% $[\Lambda][H']= f_* ([\Lambda''] \cdot (\alpha+\beta)^{a+1})$, where $\alpha$ and $\beta$ are classes of hyperplane sections of $\schemeofsections{1}$ and $\schemeofsections{3}$, respectively. Here, we used that for the class $\zeta$ of a hyperplane section of $\schemeofquartics$ we have $f^*(\zeta) = \alpha + \beta$. From the definition of $\Lambda''$ it follows that the term of highest degree of $[\Lambda'']$ is $\deg(Q')\beta^{21-a} = \deg(Q)\beta^{21-a}$



\end{proof}