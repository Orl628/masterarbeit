%!TEX root = ./master.tex
To perform calculations in the Chow ring $A$ of $\GGr(1,\schemeofsurfaces)$, we follow the conventions found in \cite{eisenbud-harris-intersection-theory}. We assume $\characteristic(k) = 0$ for simplicity. Let $N\coloneqq \dim H^0(\mc{O}(d)) = \binom{n+d}{n}$. For $N-2\geq a\geq b$, we have the Schubert cycle 
\[
	\Sigma_{a,b}\coloneqq \{T \in \GGr(1,\schemeofsurfaces) : T \cap H \neq \leer, T \subseteq H'\},
\]
where $(H\subset H')$ is a general flag of linear subspaces of dimension $N-a-2$ resp.\ $N-b-1$ in the projective space $\schemeofsurfaces$.
The ring $A$ is generated by the Schubert classes $\sigma_{a,b}$ of the cycles $\Sigma_{a,b}$.
The class $\Sigma_{a,b}$ has codimension $a+b$, and we use the convention $\sigma_{a}\coloneqq \sigma_{a,0}$.

We calculate the cohomology class of the set of jumping lines $Z$ of the Verlinde bundle $V_{d+1,t}$ of the family of hypersurfaces of degree $d+1$ in $\PP^n$, with $n\leq 3$.

% By Kleiman's theorem, finding the product of $\sigma_{a,b}$ and some class $[Z]$ of complementary codimension amounts to calculating the cardinality $\alpha$ of the set $Z\cap \Sigma_{a,b}(\mc{H})$, where $\mc{H}$ is a general flag. Then $\sigma_{a,b} \cdot [Z] = \alpha \sigma_{33,33}$.

\begin{proposition}
	Let $n \leq 3$ and let $Z$ be set of jumping lines of $V_{d+1,t}$, and let $[Z]$ be the class of $Z$ in the Chow ring $A$. Let $b$ range over the integers with the property $0\leq b < \frac{\dim Z}{2}$ and define $a=\dim Z - b, a'=a+\frac{\codim Z-\dim Z}{2}$, $b'=b+\frac{\codim Z-\dim Z}{2}$.
	\begin{enumerate}
		\huyitem If $\dim Z$ is odd or $n=2$, we have
		\begin{equation} \label{class-of-locus}
			[Z] = \sum_{a,b} \left({\binom{a+1}{n}}{\binom{b+1}{n}}-{\binom{a+2}{n}}{\binom{b}{n}}\right) \sigma_{a',b'}. 
		\end{equation}
		\huyitem If $\dim Z$ is even and $n=3$, we have
		\begin{equation*}
			[Z] = \sum_{a,b} \left({\binom{a+1}{n}}{\binom{b+1}{n}}-{\binom{a+2}{n}}{\binom{b}{n}}\right) \sigma_{a',b'}
			+
			\binom{\frac{\dim Z}{2} + 2}{n}\binom{\frac{\dim Z}{2}}{n}\sigma_{\frac{\dim Z}{2},\frac{\dim Z}{2}}.
		\end{equation*}
	\end{enumerate}
\end{proposition}

\begin{proof}
	Let $Q \subset \schemeofsurfaces$ be the image of the multiplication map $$f\from \schemeofsections{1} \times \schemeofsections{d-1} \to \schemeofsurfaces$$ as in \Cref{helping-varieties}. The map $f$ is birational on its image, since a general point of $Q$ has the form $gh$ with $h$ irreducible. By
	\Cref{description-jumping-lines-as-reduction}, the variety $Z$ is the image of the finite multiplication map
	$$\phi \from \GGr(1,\schemeofsections{1}) \times \schemeofsections{d-1} \to \GGr(1,\schemeofsurfaces).$$

	%We have $\dim(Q) = 22$ and $\dim(Z)=23$, while $\codim(Z)=43$. 
	The Chow group $A^{\codim Z}$ is generated by the classes $\sigma_{a',b'}$ with $N-2\geq a'\geq b' \geq \floor{\frac{\codim Z}{2}}$ and $a'+b'=\codim Z$, while the complementary group $A^{\dim Z}$ is generated by the classes
	$\sigma_{\dim Z-b,b}$ with $b\in {0,\dotsc,\floor{\frac{\dim Z}{2}}}$. Write
	\[
		[Z] = \sum_{a',b'} \alpha_{a',b'} \sigma_{a',b'}. 
	\]
	We have $\sigma_{a',b'} \sigma_{a,b} = 1$ if $b'-b = \floor{\frac{\codim Z}{2}}$ and $0$ else. Hence, multiplying the above equation with the complementary classes $\sigma_{a,b}$ and taking degrees gives
	$
	\alpha_{a',b'} = \deg([Z]\cdot \sigma_{a,b}).
	$

	Using Giambelli's formula 
	$\sigma_{a,b}=\sigma_{a}\sigma_{b} - \sigma_{a+1}\sigma_{b-1}$ \cite[Prop.\ 4.16]{eisenbud-harris-intersection-theory}, we reduce to computing $\deg([Z] \cdot \sigma_{a}\sigma_{b})$ for $0\leq b\leq \floor{\frac{\dim Z}{2}}$.
	By Kleiman transversality, we have 
	\[
	\deg([Z] \cdot \sigma_{a}\sigma_{b}) = \abs{\{T\in Z : T \cap H \neq \leer, T \cap H' \neq \leer\}},
	\]
	where $H$ and $H'$ are general linear subspaces of $\schemeofsurfaces$ of dimension $N-a-2$ and $N-b-2$, respectively.

	To a point $p = g_p h_p \in Q$ with $g_p \in \schemeofsections{1}$ and $h_p \in \schemeofsections{d-1}$, associate a closed reduced subscheme $\Lambda_p\subset Q$ containing $p$ as follows. If $h_p$ is irreducible, let $\Lambda_p$ be the image of the linear embedding $\schemeofsections{1}\times \{h_p\} \to \schemeofsurfaces$ given by $g \mapsto g h_p$.

	If $h_p$ is reducible, define the subscheme $\Lambda_p$ as the union $\bigcup_h \im(\schemeofsections{1} \times \{h\}\to \schemeofsurfaces)$, where $h$ ranges over the (up to multiplication by units) finitely many divisors of $p$ of degree $d-1$.

	Note that for all points $p$, the spaces $\im(\schemeofsections{1} \times \{h\}\to \schemeofsurfaces)$ meet exactly at $p$.

	% If $h_p = g'_p h'_p$ with $h'_p \in \schemeofsections{2}$ irreducible, let $\Lambda_p$ be the union of the images of the linear embeddings $\schemeofsections{1} \times \{h_p\} \to \schemeofsurfaces$ and $\schemeofsections{1} \times \{g_p h'_p\} \to \schemeofsurfaces$. These two linear subspaces meet exactly at $p$. Similarly, if $p$ is the product of four linear forms, define the space $\Lambda_p$ as the union $\bigcup_h \im(\schemeofsections{1} \times \{h\}\to \schemeofsurfaces)$, where $h$ runs over the four cubics arising as products of the linear factors of $p$.

	By the definition of $Z$, all lines $T\in Z$ lie in $Q$. Furthermore, if $T$ meets the point $p$, then $T\subseteq \Lambda_p$.
	For $H\subseteq\schemeofsurfaces$ a linear subspace of dimension $N-a-2$, define $Q'\coloneqq H\cap Q$. For general $H$, the subscheme $Q'$ is a smooth subvariety of dimension $b-n+1$ such that for a general point $p=gh$ of $Q'$ with $h\in \schemeofsurfaces$, the polynomial $h$ is irreducible.

	% By the construction of the $\Lambda_p$, if a line $T\in Z$ meets the point $p$, then $T\subseteq \Lambda_p$. A general linear subspace $H$ of dimension $10+a$ intersects the variety $Q$ at a smooth subvariety $Q'$ of dimension $a-2$. Furthermore, it may be assumed that a general point $p=gh$ of $Q'$ with $h\in \schemeofsections{3}$ has $h$ irreducible. Indeed, the subvariety of $Q$ where $h$ is reducible has dimension $\dim(\schemeofsections{1} \times \schemeofsections{1} \times \schemeofsections{2}) = 15$, so the dimension of its intersection with a general $H$ is $a-9$.

	Next, we consider the case $n=2$ or $\dim Z$ odd. We show that for general $H$, for each point $p\in Q'$ we have $\Lambda_p \cap H =\{p\}$. Let $\mc{H}$ denote the parameter space for $H$, \ie the Grassmannian $\Gr(\dim H+1, N)$. Define the closed subset $X\subseteq Q\times \mc{H}$ by
	$X\coloneqq \{(p,H):\dim(H\cap \Lambda_p)\geq 1\}$. The fibers of the induced map $X\to \mc{H}$ have dimension at least one. Hence, to prove that the desired condition on $H$ is an open condition, it suffices to prove $\dim(X) \leq \dim(\mc{H})$. The fiber of the map  $X\to Q$ over a point $p$ consists of the union of finitely many closed subsets of the form $X'_p = \{H\in \mc{H} : \dim(H\cap \Lambda'_p)\geq 1\}$, where $\Lambda'_p\isom \PP^n\subseteq \schemeofsurfaces$ is one of the components of $\Lambda_p$. The space $X'_p$ is a Schubert cycle
	\[
		\Sigma_{\dim Q - b,\dim Q - b} = \{H\in \Gr(\dim H+1, N) : \dim(H \cap H_{n+1}) \geq 2\},
	\]
	with $H_{n+1}$ an $(n+1)$-dimensional subspace of $H^0(\mc{O}(d))$. The codimension of the cycle is $2(\dim Q - b)$, hence also $\codim(X_p) = 2(\dim Q -b)$. Finally, we have $\dim(\mc{H})-\dim(X) = \codim(X_p) - \dim(Q) = \dim Q - 2b.$
	If $\dim Z$ is odd, then $\dim Q - 2b \geq \dim Q - \dim Z + 1 = 3-n\geq 0$. If $n=2$, we instead estimate $\dim Q - 2b \geq \dim Q - \dim Z = 2-n\geq 0$.

	Next, let $\Lambda \coloneqq \bigcup_{p\in Q'} \Lambda_p = f(\schemeofsections{1}\times \pr_2 f^{-1}(Q'))$ and $\Lambda''\coloneqq \schemeofsections{1}\times \pr_2 f^{-1}(Q')$. By the choice of $H$, the map $f^{-1}(Q')\to Q'$ is birational and the map $f^{-1}(Q')\to \pr_2f^{-1}(Q')$ is even bijective. It follows that $\Lambda''$ and hence $\Lambda$ have dimension $b+1$. The intersection of $\Lambda$ with a general linear subspace $H'$ of dimension $N-b-2$ is a finite set of points. For each point $p\in Q'$, the linear subspace $H'$ intersects each component $\Lambda'_p$ of $\Lambda_p$ in at most one point. For each point $p'\in H'\cap\Lambda$ there exists a unique $p$ such that $p'\in\Lambda_p$. Furthermore, the only line $T\in Z$ meeting both $p$ and $H'$ is the one through $p$ and $p'$. If the intersection $H'\cap \Lambda_p$ is empty, then there will be no line meeting $p$ and $H'$. Hence, $\deg([Z]\cdot \sigma_{a}\sigma_{b})$ is the number of intersection points of $\Lambda$ with a general $H'$.

	Finally, the pre-image $f^{-1}(Q') = f^{-1}(H)$ is smooth for a general $H$ by Bertini's Theorem. If $\zeta$ is the class of a hyperplane section of $\schemeofsurfaces$ we have $f^*(\zeta) = \alpha + \beta$, , where $\alpha$ and $\beta$ are classes of hyperplane sections of $\schemeofsections{1}$ and $\schemeofsections{d}$, respectively. Since $\pr_2$ and $f$ have degree one, we compute:
	\begin{align*}
	[\Lambda''] & = [\pr_2^{-1}\pr_2 f^{-1}(H)] \\
	& = \pr_2^*[\pr_2 f^{-1}(H)] \\
	& = \pr_2^* \pr_{2,*}[f^{-1}(H)] \\
	& = \pr_2^*\pr_{2,*}f^*[H] \\
	& = \pr_2^* \pr_{2,*}(\alpha + \beta)^{\codim H} \\
	& = \textstyle{\binom{\codim H}{n}}\pr_2^* \beta^{\codim H -n} \\
	& = \textstyle{\binom{\codim H}{n}}\beta^{\codim H -n}
	\end{align*}

	Hence, by the push-pull formula:
	\begin{align*}
		\deg([\Lambda]\cdot H') &= \deg([\Lambda'']\cdot(\alpha+\beta)^{\codim H'}) \\ &= \binom{\codim H}{n}\binom{\codim H'}{n}
		\\ &=\binom{a+1}{n}\binom{b+1}{n}.
	\end{align*}
	We then use Giambelli's formula to obtain Equation \cref{class-of-locus}.

	In case $n=3$ and $\dim Z$ even, we need to show that for $b=\dim Z/2$ we have
	$\deg ([Z]\cdot \sigma_{b,b})=0$. In this case, the hyperplanes $H$ and $H'$ have the same dimension $N-b-2$.

	For $p\in Q$, the set $\Lambda_p$ is defined as before. We now claim that for general $H$ of dimension $N-b-2$, we have
	$\dim(\Lambda_p\cap H)=1$. Consider as before the closed subscheme
	$
		X\coloneqq \{(p,H) : \dim(\Lambda_p\cap H)\geq 1\}
		\subset Q\times \mc{H}.
	$
	The generic fiber of the projection map $\phi\from X\to \mc{H}$ is one-dimensional, hence we have
	$\dim \phi(X) = \dim(X)-1 = \dim \mc{H}$. The last equation holds with $n=3$ and $2b=\dim Z$. Hence for all $H\in \mc{H}$ we have $\dim (\Lambda_p \cap H)\geq 1$.

	On the other hand, the equality $\dim(\Lambda_p \cap H)=1$ is attained by some, and hence by a general, $H$. Indeed, consider the closed subscheme
	$
		\wtilde X \coloneqq \{(p,H) : \dim(\Lambda_p\cap H)\geq 2\}
		\subset Q\times \mc{H}.
	$
	By a similar argument as before, one needs to show that
	$\dim(\mc{H})-\dim(X) + 1\geq 0$. The fiber $X_p$ is a Schubert cycle of codimension
	$3(\dim Q-b+1)$. Lastly, a computation shows
	$\dim(\mc{H})-\dim(\wtilde X)+1=\codim(\wtilde X_p)-\dim(Q)+1=\frac{1}{2}(2\dim Q + 18 - 5n)\geq 0$.

	Now, define $\Lambda''$ as above. We have
	$\dim \Lambda''
	= \dim \schemeofsections{1} + \dim \pr_2 f^{-1}(Q')
	= b$.
	Since $f$ is generically of degree one, we still have $\dim \Lambda'' = \Lambda$, hence $\dim \Lambda + \dim H' = N-2 < \dim \schemeofsurfaces$. It follows that a generic $H'$ does not meet any of the lines $T\subset Z$, hence $\sigma_{b}\sigma_{b}\cdot [Z] = 0$.

	% To compute this number, we compute the coefficient $\lambda$ of the term of highest degree of the class
	% $[\Lambda][H']= f_* ([\Lambda''] \cdot (\alpha+\beta)^{a+1})$, where $\alpha$ and $\beta$ are classes of hyperplane sections of $\schemeofsections{1}$ and $\schemeofsections{3}$, respectively. Here, we used that for the class $\zeta$ of a hyperplane section of $\schemeofsurfaces$ we have $f^*(\zeta) = \alpha + \beta$. From the definition of $\Lambda''$ it follows that the term of highest degree of $[\Lambda'']$ is $\deg(Q')\beta^{21-a} = \deg(Q)\beta^{21-a}$
\end{proof}
