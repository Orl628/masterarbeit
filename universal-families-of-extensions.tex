\section{Universal Families of Extensions}

Let $X$ and $S$ be Noetherian schemes over a field $k$. Let $f \from X \to S$ be a flat, projective morphism, and let $\mc{F}$ and $\mc{G}$ be coherent $\mc{O}_X$-modules, flat over $\mc{O}_X$.

Recall that an element $\xi \in \Ext^1_{X}(\mc{F},\mc{G})$ corresponds to an equivalence class of short exact sequences, or \emph{extensions}, of the form
\[
	\ses{\mc{G}}{\mc{E}}{\mc{F}},
\]
where two such sequences are equivalent if there exists an isomorphism between them that induces the identity on $\mc{F}$ and $\mc{G}$. The set of these equivalence classes can be given the structure of an $H^0(S,\mc{O}_S)$-module, see for example \cite[{}3.4]{weibel-homological-algebra}. This correspondence is functorial in both arguments, and preserves the $H^0(S, \mc{O}_S)$-module structure.

Explicitely, the sum of two elements of $\Ext^1$ corresponds to the Baer sum of the associated extensions, while the multiplication of an extension
as above by a scalar $a\in H^0(S,\mc{O}_S)$ is given by the pullback sequence along the map $\mc{F}\xto{a}\mc{F}$.

In this section, we ask when it is possible to construct an $S$-scheme $V$ and a universal extension on $X\times_S V$. This is quickly found to be true for $\mc{F},\mc{G}$ locally free and $S=\Spec(k)$. For the more general situation, the article \cite{lange-universal-extensions} turns to the moduli problem of classifying relative extensions of sheaves and applies it to the global situation.
%The next proposition shows there exists a $k$-scheme $V$ that parametrizes the points of $\Ext^{1}_X(\mc{F},\mc{G})$.
\begin{remark}
Let $\phi \from X_1 \to X_2$ be a morphism of schemes, let $\mc{F}_1$ and $\mc{F}_2$ be $\mc{O}_{X_1}$-modules. The Grothendieck spectral sequence \cite[Theorem~23.3.5]{vakil-algebraic-geometry} specializes to the Leray spectral sequence
$E_2^{p,q}=H^q(X_2,R^p \phi_*\mc{F}_1) \Rightarrow H^{p+q}(X_1,\mc{F}_1)$
%\cite[Theorem~23.4.5]{vakil-algebraic-geometry}
and the local-to-global $\Ext$ spectral sequence
$E_2^{p,q}=H^p(X_1,\EXT^q(\mc{F}_1,\mc{F}_2)) \Rightarrow \Ext^{p+q}(\mc{F}_1,\mc{F}_2)$.
The first few terms of the associated exact sequences in lower degrees are 
\begin{equation}
	0
	\to H^1(X_2,\pushf \phi \mc{F}_1)
	\to H^1(X_1,\mc{F}_1)
	\to H^0(X_2, R^1 \pushf \phi \mc{F}_1) \label{leray-es}
\end{equation}
and
\begin{equation}
	0
	\to H^1(X_2,\HOM(\mc{F}_1,\mc{F}_2))
	\to \Ext^1(\mc{F}_1,\mc{F}_2)
	\to H^0(X_2,\EXT^1(\mc{F}_1,\mc{F}_2)). \label{ext-es}
\end{equation}
\end{remark}

\begin{proposition} \label{cor:universal-extension}
Let $\mc{F}$ and $\mc{G}$ be locally free, and let $V \coloneqq \VV(\Ext^1_X(\mc{F},\mc{G})\dual)$. There exists an extension
	\[
		\xi_{\mathrm{univ}} \colon \quad \ses{\pullb{\pr_1}\mc{G}}{\mc{E}}{\pullb{\pr_1}\mc{F}}
	\]
over $X\times_k V$
such that for all Noetherian $k$-schemes $Y$, the map
$$\Mor_{k}(Y,V)\to\Ext^1_{X_Y}(\mc{F}_Y,\mc{G}_Y)$$ defined by
$\alpha \mapsto \pullb{(\id_X\times \alpha)} \xi_{\mathrm{univ}}$
is a bijection, functorial in $Y$.
In particular, pulling back $\xi_{\text{univ}}$ gives a bijection $\Mor_{k}(\Spec(k),V)\xto{\sim} \Ext^1_{X}(\mc{F},\mc{G})$.
\end{proposition}

\begin{proof}
We find functorial isomorphisms
\begin{align}
\Mor_{k}(Y,V)
% & \simeq \Hom_{\text{k-mod}}(\Ext^1_X(\mc{F},\mc{G})\dual,A) \label{ext-isom-univprop}\\
%\simeq & A \otimes_k \Ext^1_{X}(\mc{F},\mc{G}).
& \simeq H^0(\mc{O}_Y \otimes \Ext^1(\mc{F},\mc{G})) \label{isom-univprop} \\
& \simeq H^0(s_{2}^* H^1(\HOM(\mc{F},\mc{G}))) \label{isom-ext-ss} \\
& \simeq H^0(R^1 \pr_{2,*}(\pr_1^* \HOM(\mc{G},\mc{F}))) \label{isom-cbc} \\
& \simeq H^0(R^1 \pr_{2,*}(\HOM(\mc{F}_Y,\mc{G}_Y))) \label{isom-locfree}\\
& \simeq H^1(\HOM(\mc{G}_Y,\mc{F}_Y)) \label{isom-leray-ss-via-cbc}\\
& \simeq \Ext^1_{X_Y}(\mc{F}_Y,\mc{G}_Y) \label{isom-ext-ss-two}.
\end{align}
The required universal extension is then the image of $\id\in\Mor_{k}(V,V)$ in $\Ext^1_{X_Y}(\mc{F}_Y,\mc{G}_Y)$.

The isomorphism \cref{isom-univprop} comes from the universal property of $\VV(\Ext^1_X(\mc{F},\mc{G})\dual)$. The isomorphisms \cref{isom-ext-ss} and \cref{isom-ext-ss-two} come from the sequence \cref{ext-es}, whose third term is zero since $\mc{F}$ and $\mc{G}$ are locally free. We have \cref{isom-cbc} by the Cohomology and Base Change Theorem \cite[{}28.1.6]{vakil-algebraic-geometry}, and \cref{isom-locfree} since $\mc{F}$ and $\mc{G}$ are locally free. For the isomorphism \cref{isom-leray-ss-via-cbc}, we use the sequence \cref{leray-es}, whose third term is found to be zero after applying the Cohomology and Base Change Theorem.
\end{proof}

\begin{definition}\begin{enumerate}
	\item The \emph{$i$-th relative Ext module} $\EXT^i_f(\mc{F},\mc{G})$ is the image of $\mc{G}$ under the $i$-th right-derived functor of
	$\pushf{f}\HOM(\mc{F},-)
	\from \Mod_{\mc{O}_X} \to \Mod_{\mc{O}_S}$.

	\item For $s\in S$, define the homomorphism
	\[
		\Phi_s = \Phi_{s,\mc{F},\mc{G}}
		\from \Ext^1_X(\mc{F},\mc{G})
		\to \Ext^1_{X_s}(\mc{F}_s, \mc{G}_s)
	\]
	by restricting extensions of $\mc{F}$ by $\mc{G}$ to the fiber $X_s$. This is well-defined, since $\mc{F}$ is flat over $S$.

	\item A \emph{family of extensions} of $\mc{F}$ by $\mc{G}$ over $S$ is a family
	\[
		\xi_s \in \Ext^1_{X_s}(\mc{F}_s,\mc{G}_s),\quad s\in S
	\]
	such that there exists an open covering $\mathfrak{U}$ of $S$ and for all $U\in \mathfrak{U}$ an extension $\xi_U \in \Ext^1_{f^{-1}(U)}(\mc{F}_U,\mc{G}_U)$
	with $\Phi_{s,\mc{F}_U,\mc{G}_U}(\xi_U) = \xi_s$ for all $s\in S$.
	Such a family is \emph{globally defined} if we can take $\mathfrak{U} = \{S\}$.
\end{enumerate}
\end{definition}

\begin{remark}
	If $S$ is affine, then we have $\EXT^i_f(\mc{F},\mc{G})=\Ext^i_X(\mc{F},\mc{G}){\ \widetilde{}}$.
\end{remark}


% \begin{proposition}
% 	Let $g \from Y \to S$ be a morphism of Noetherian schemes.
% 	There exists a number $N\geq 0$ depending on $\mc{G}$ such that for all quasi-coherent $\mc{O}_Y$-modules $\mc{M}$, all $i\geq 1$ and $n\geq N$ we have
% 	\[
% 		\EXT^i_{f_Y}(\mc{O}_{X_Y}(-n),\mc{G} \boxtimes \mc{M}) = 0
% 	\]
% \end{proposition}

\begin{proposition}
	Let $g \from Y \to S$ be a morphism of Noetherian schemes.
	For all $i\geq 0$ there exists a canonical base change homomorphism
	\[
		\tau^i_{g} \from \pullb{g}\EXT^i_{f}
		(\mc{F},\mc{G})
		\to
		\EXT^i_{f_Y}
		(\pullb{g_X}\mc{F}, \pullb{g_X}\mc{G}).
	\]
	Furthermore, if $g$ is flat, then $\tau^i_g$ is an isomorphism for all $i\geq 0$.
\end{proposition}

\begin{proof}
	See \cite[Prop. 1.3]{lange-universal-extensions}
\end{proof}

\begin{definition}
	We say that $\EXT^i_{f}(\mc{F},\mc{G})$ \emph{commutes with base change} if for all morphisms of Noetherian schemes $g \from Y \to S$, the base change homomorphism $\tau^i_g$ is an isomorphism.
\end{definition}

\begin{proposition}
\label{prop:ext-base-change}
	Let $s \in S$ be a point such that $\tau^i_s$ is surjective. Then there exists an open neighborhood $U$of $s$ such that $\tau^i_{s'}$ is an isomorphism for all $s'\in U$.
	Furthermore, the homomorphism $\tau^{i-1}_{s}$ is surjective if and only if $\EXT^i_{f}(\mc{F},\mc{G})$ is locally free on an open neighborhood of $s$.
\end{proposition}

\begin{proof}
	See \cite[Thm. 1.4]{lange-universal-extensions}
\end{proof}

\begin{remark}
\begin{enumerate}
	\item If $\tau^i_{s}$ is an isomorphism for all $s\in S$, then $\EXT_{f}^i(\mc{F},\mc{G})$ commutes with base change.

	\item From \Cref{prop:ext-base-change} we conclude that if $\EXT^i_f(\mc{F},\mc{G})$ commutes with base change for $i=0,1$, then $\EXT^1_f(\mc{F},\mc{G})$ is locally free.

	\item In case $S$ is reduced, if $\EXT^1_f(\mc{F},\mc{G})$ is locally free then $\EXT^i_f(\mc{F},\mc{G})$ commutes with base change for $i=0,1$.
\end{enumerate}
\end{remark}

\begin{definition} 
	Let $u\from Y'\to Y$ be a morphism of Noetherian $S$-schemes.

	\begin{enumerate}\item We define a functoriality map
	$H^0(Y,\EXT^1_{f_Y}(\mc{F}_Y,\mc{G}_Y))
	\to
	H^0(Y',\EXT^1_{f_{Y'}}(\mc{F}_{Y'},\mc{G}_{Y'}))
	$
	as the composition
	\begin{align*}
		H^0(Y,\EXT^1_{f_Y}(\mc{F}_Y,\mc{G}_Y))
		\xto{1\otimes \id} & 
		H^0(Y',\pullb{u}\EXT^1_{f_Y}(\mc{F}_Y,\mc{G}_Y)) \\
		\xto{H^0(\tau^1_u)} &
		H^0(Y',\EXT^1_{f_{Y'}}(\pullb{u_{X_Y}}\mc{F}_{Y'},\pullb{u_{X_Y}}\mc{G}_{Y'})).
	\end{align*}

	\item Given a family of extensions $\xi = (\xi_y)_{y\in Y}$ of $\mc{F}_Y$ by $\mc{G}_Y$ over $Y$, we set $(\pullb{u}\xi)_{y'} \coloneqq \pullb{u}\xi_{u(y')}$ for every $y' \in Y'$. This defines a family $\pullb{u}\xi$ of extensions of $\mc{F}_{Y'}$ by $\mc{G}_{Y'}$ over $Y'$.

	\item We define the functors
	\begin{align*}
		& E,E' \from (\textup{NoethSch}/S) \to (\textup{Sets}); \\
		%E'_{\text{glob}}
		& E(Y) \coloneqq H^0(Y,\EXT^1_{f_Y}(\mc{F}_Y,\mc{G}_Y)), \\
		& E'(Y) \coloneqq \{\text{families of extensions of $\mc{F}_Y$ by $\mc{G}_Y$ over } Y\}.
%		& E'_{\glob}(Y) \coloneqq \{\text{globally defined families of extensions of $\mc{F}_Y$ by 
%		$\mc{G}_Y$ over } Y\}.
	\end{align*}
\end{enumerate}
\end{definition}
\begin{remark}
	The Grothendieck spectral sequence for the sequence of functors
	\[\Mod_{\mc{O}_X}\xto{\HOM(\mc{F},-)} \Mod_{\mc{O}_X}\xto{\pushf f} \Mod_{\mc{O}_S}\]
	is the spectral sequence with $E_2^{p,q}=H^p(S,\EXT^q_f(\mc{F},\mc{G})) \Rightarrow \Ext^{p+q}_X(\mc{F},\mc{G})$. This gives the exact sequence
	\begin{align*}
		0
		 \to & H^1(S,\pushf{f}\HOM(\mc{F},\mc{G}))
		 \xto{\varepsilon} \Ext^1_X(\mc{F},\mc{G})
		 \xto{\mu} H^0(S,\EXT^1_f(\mc{F},\mc{G})) \\
		 \xto{d_2} & H^2(S,\pushf{f}\HOM(\mc{F},\mc{G})). \label{es-rel-ext} \addtocounter{equation}{1}\tag{\theequation}
	\end{align*}
\end{remark}

\begin{proposition}
	Suppose that $S$ is reduced and $\EXT^1_f(\mc{F},\mc{G})$ commutes with base change. Restricted to the category of reduced Noetherian $S$-schemes, the functors $E$ and $E'$ are isomorphic.
	% Under this isomorphism, the globally defined extensions correspond to the subgroup
	% $\Ext^1_{X_Y}(\mc{F}_Y,\mc{G}_Y)/H^1(Y,f_{Y,*} \HOM_{X_Y}(\mc{F}_Y,\mc{G}_Y)) \subseteq H^0(Y,\EXT^1_{f_Y}(\mc{F}_Y,\mc{G}_Y)$.
\end{proposition}
\begin{proof}
	See \cite[Prop. 2.3]{lange-universal-extensions}.
\end{proof}
\begin{proposition} \label{e-representable}
	Suppose that $\EXT^i_f(\mc{F},\mc{G})$ commutes with base change for $i=0,1$. Then the $\mc{O}_S$-module $\EXT^1_f(\mc{F},\mc{G})\dual$ is locally free and the functor $E$ is representable by the $S$-scheme $\VV(\EXT^1_f(\mc{F},\mc{G})\dual)$.
\end{proposition}
\begin{proof}
	See \cite[Prop. 3.1]{lange-universal-extensions}.
\end{proof}
\begin{corollary}
	Suppose that $S$ is reduced and $\EXT^i_f(\mc{F},\mc{G})$ commutes with base change for $i=0,1$. Restricted to the category of reduced Noetherian $S$-schemes, the functor $E'$ is representable by the $S$-scheme $\VV(\EXT^1_f(\mc{F},\mc{G})\dual)$.
\end{corollary}

\begin{corollary}
	Suppose that for all Noetherian $S$-schemes $Y$ we have
	$$H^i(Y,f_{Y,*} \HOM_{X_Y}(\mc{F}_Y,\mc{G}_Y)) = 0$$ for $i=1,2$. The functor $Y \mapsto \Ext^1_{X_Y}(\mc{F}_Y,\mc{G}_Y)$ is representable by the $S$-scheme $\VV(\EXT^1_f(\mc{F},\mc{G})\dual)$.
\end{corollary}

\begin{proof}
	Use the sequence \cref{es-rel-ext} and \Cref{e-representable}.
\end{proof}

% \begin{corollary}
% 	Suppose that $S$ is affine and $\EXT^1_f(\mc{F},\mc{G})$ commutes with base change for $i=0,1$. The functor
% 	\[
% 		(\textup{Aff}/{S}) \to (\textup{Sets})
% 		\colon
% 		Y \mapsto \Ext^1_{X_Y}(\mc{F}_Y,\mc{G}_Y)
% 	\]
% 	is representable by the $S$-scheme $\VV(\EXT^1_f(\mc{F},\mc{G})\dual)$.
% \end{corollary}

\begin{remark}
As a special case of the above, we recover \Cref{cor:universal-extension}.
\end{remark}

\begin{remark}
	The article \cite{lange-universal-extensions} continues on to define a (projectivized) version of the problem, so that over $\Spec(k)$, the scheme $\PP(\Ext^1_X(\mc{F},\mc{G})\dual)$ parametrizes the equivalence classes of nonsplit extensions of $\mc{F}$ by $\mc{G}$, modulo the action of $k^{\times}$. See also \cite[Example 2.1.12]{huybrechts-lehn-sheaves}.
\end{remark}

% \begin{proposition}
% 	 Let $\basescheme$ be reduced and suppose that $\EXT^1_\totalmap(\extendedtotalmodule,\extendingtotalmodule)$ commutes with base change. Then there exists a bijection, functorial in $Y$ between the set of all families of extension of $\extendedtotalmodule$ by $\extendingtotalmodule$over $Y$ and the set
% 	 $H^0(\basescheme, \Ext^1_\totalmap(\extendedtotalmodule, \extendingtotalmodule))$.
% \end{proposition}