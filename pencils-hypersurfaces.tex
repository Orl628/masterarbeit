\section{Verlinde Bundles on Pencils of Hypersurfaces}

\begin{definition}
Let $\pi\from \mf X \to \abs{\mc{O}_{\PP^n}(d)}$ be the universal family of hypersurfaces of degree $d$ in $\PP^n$. Let $\mc{L}$ be the restriction of the bundle $\mc{O}(1)\boxtimes \mc{O}$ under the inclusion $\mf X \subseteq \PP^n \times \schemeofsurfaces$. The vector bundle $V_k \coloneqq \pushf \pi (\mc{L}^{\otimes k})$ is the $k$-th \emph{Verlinde Bundle} of the universal family $\pi$.
\end{definition}
The vector bundle $V_k$ is the cokernel of the map
\[
M \from \mc{O}_{\schemeofsurfaces}(-1)
\otimes H^0(\PP^n,\mc{O}(k-d))
	\to \mc{O}_{\schemeofsurfaces} \otimes H^0(\PP^n,\mc{O}(k))
\]
given by multiplication by
$\sum_I \alpha_I \otimes x^I
\in H^0(\schemeofsurfaces,\mc{O}(1)) \otimes H^0(\PP^n,\mc{O}(d)),$ where the $\alpha_I$ are the homogeneous coordinates on $\schemeofsurfaces$. The vector bundle $V_k$ has rank $r^{(k)}\coloneqq h^0(\PP^n, \mc{O}(k))-h^0(\PP^n, \mc{O}(k-d))$ and determinant $\mc{O}(d^{(k)})$ with $d^{(k)}\coloneqq h^0(\PP^n,\mc{O}(k-d))$.
We study the restriction of $V_k$ to lines $T\subseteq \schemeofsurfaces.$

% \begin{proposition} \label{construct-splitting}
% 	Let $f_1, f_2 \in \schemeofsurfaces$ span the line
% 	$T \subseteq \schemeofsurfaces$. 
% 	The map $M|_{T}$ splits as a direct sum
% 	\[
% 		\bigoplus_{i=1}^s M_i
% 		\from \bigoplus_{i=1}^s \mc{F}_i
% 		\to \bigoplus_{i=0}^s \mc{G}_i
% 	\]
% 	with $\mc{F}_i \isom \mc{O}(-1)^{\lambda_i}$,
% 	$\mc{G}_i \isom \mc{O}^{\lambda_i + 1}$ for $i\geq 1$, and $\mc{G}_0 \isom \mc{O}^{\lambda_0}$. This splitting gives rise to a splitting of the cokernel
% 	$\coker(M|_{T}) = \mc{O}^{\lambda_0}
% 	\oplus \bigoplus_{i=1}^s \mc{O}(\lambda_i)$.
% \end{proposition}

% \begin{proof}[Proof (Construction)]

% 	Start with 

%  	We construct $M_1 \from \mc{F}_1 \to \mc{G}_1$, the others can be defined similarly. The bundle $\mc{G}_0$ will then just be a direct complement of $\bigoplus_{i\geq 1}\mc{G}_i$. Note that for all $\theta \in H^0(\mc{O}(k-d), \PP^n),$ the morphism $M|_T$ maps $\mc{O}(-1) \otimes \gen{\theta}$ into $\mc{O} \otimes \gen{f_1 \theta , f_2\theta }$.

%  	The bundle $\mc{F}_1$ shall have the form $\mc{O}(-1) \otimes U_1$, with $U_1$ a subspace of $H^0(\PP^n,\mc{O}(k-d))$, while $\mc{G}_1 = \mc{O} \otimes W_1$, with $W_1 \coloneqq f_1 U_1 + f_2 U_1$. We construct $U_1$ iteratively, beginning with $U_1^{0} \coloneqq \gen{\theta}$, where $\theta$ is any nonzero section. After $U_1^{\nu}$ is constructed, we define $W_1^{\nu} \coloneqq f_1 U_1^{\nu} + f_2 U_1^{\nu}$ and either stop or construct $U_1^{\nu + 1}$ as follows.

%  	If the space $f_1 U_1^{\nu,\perp} + f_2 U_1^{\nu,\perp}$ does not intersect $W_1^\nu$, we stop and set $U_1 \coloneqq U_1^\nu$. In the other case, there exist $\theta'_1, \theta'_2 \in U_1^{\nu, \perp}$ with $f_1 \theta'_1 + f_2 \theta'_2 \in W_1^{\nu}.$ If $\theta'_2 \in U_1^{\nu,\perp}$, we set $U_1^{\nu+1} \coloneqq U_1 + \gen{\theta'_1}$. Otherwise, We set $U_1^{\nu+1} \coloneqq U_1^{\nu} + \gen{\theta'_1,\theta'_2}$.

%  	At the end of the construction, we may split $M_1 \from \mc{F}_1 \to \mc{G}_1$ as a direct summand of $M|_T$, and perform the same construction for the remaining $M_i$.


 	% If no other subbundle $\mc{O}(-1)\otimes \gen{\theta_j}$ maps into $\mc{O} \otimes \gen{f_1\theta_1,f_2\theta_1}$ we stop, so that $U_1 = \gen{\theta_1}$. Else we pick such a basis element $\theta_j$, and add it to $U_1$. We iterate this process until $f_1 U_1^\perp + f_2 U_1^\perp$ does not intersect $W_1$. At that point, we may split $M_1 \from \mc{F}_1 \to \mc{G}_1$ as a direct summand of $M|_T$.

%  	Note that after each iterative step in the construction, apart from the first, the dimension of $W_1$ could increase either by zero or by one. To show that only the second case occurs, we exhibit a linearly independent set of size $j+1$ in ${\gen{f_1 \theta_1, f_2 \theta_1, \dotsc, f_1\theta_{j}, f_2\theta_{j}}}$. Here, the indices of the $\theta_{j'}$ are simplified for convenience. Having shown this, it follows that $\rank{\mc{G}_1} = \lambda_1 + 1$.

%   Consider an ordering of the monomial basis of $H^0(\mc{O}(1),\PP^n)$ and accordingly order the monomial bases of all $H^0(\mc{O}(m),\PP^n)$ lexicographically. Let $\theta_{\max}$ denote the greatest basis element among the $(\theta_{j'})_{j'=1}^j$. Assume without loss of generality that the greatest monomial appearing in $f_2$ is strictly greater than all monomials appearing in $f_1$. Then $(f_1\theta_1,\dotsc,f_1\theta_j,f_2\theta_{\max})$ is a linearly independent set.

%   Finally, the direct summands $\coker(M_i)$ of the locally free sheaf $\coker(M)$ are themselves locally free, and their form can be found by comparing ranks and determinants.
% \end{proof}

\begin{lemma} \label{mini-splitting-lemma}
	Let $\mc{E}$ be a finite free $\mc{O}_{\PP^1}$-module, and let 
	\[
		0 \to \mc{E}' \xrightarrow{\phi} \mc{E} \to \mc{E}'' \to 0
 	\]
 	be a short exact sequence of $\mc{O}_{\PP^1}$-modules. Given a splitting $\mc{E}'' = \mc{E}''_1 \oplus \mc{O}$, we may construct a splitting $\mc{E} = \mc{E}_1 \oplus \mc{O}$ such that the image of $\phi$ is contained in $\mc{E}_1$.
\end{lemma}

\begin{proof}
	Define $\mc{E}_1 \coloneqq \ker(\pr_2\circ\psi)$, which is a locally free sheaf on $\PP^1$. By comparing determinants in the short exact sequence $0 \to \mc{E}_1 \to \mc{E} \to \mc{O} \to 0$ we see that $\mc{E}_1$ is free, hence by an $\Ext^1$ computation the sequence splits. The property $\im(\phi) \subseteq \mc{E}_1$ follows from the definition.
\end{proof}

\begin{proposition} \label{number-zeroes}
	Let $f_1, f_2 \in \schemeofsurfaces$ span the line
	$T \subseteq \schemeofsurfaces$ and $\coker(M|_{T}) \isom \mc{O}^{\lambda_0}
	\oplus \bigoplus_{i=1}^s \mc{O}(d_i)$. Define $U\coloneqq H^0(\PP^n,\mc{O}(k-d))$. We have $$d_0 = \dim H^0(\PP^n,\mc{O}(k)) - \dim ({f_1 U + f_2 U}),$$
	or, equivalently,
	\[
		s = \dim({f_1 U + f_2 U}) - d^{(k)}.
	\]
\end{proposition}

\begin{proof}
	Note that the map $M|_T$ sends a local section $\xi \otimes \theta$ to $s\xi \otimes f_1 \theta + t\xi \otimes f_2 \theta$. In particular, the image of $\mc{O}(-1)\otimes U$ is contained in $\mc{O} \otimes (f_1 U + f_2 U)$. It follows that $d_0 \geq \dim ({f_1 U + f_2 U})$.

  To prove the other inequality, consider the induced sequence
  \[
  	0 \to \mc{O}(-1)\otimes U \xrightarrow{M|_T} \mc{O}\otimes (f_1 U + f_2 U) \to \mc{E}'' \to 0
  \]
  and assume that $\mc{E}'' \isom \mc{E}_1''\oplus \mc{O}.$ By \Cref{mini-splitting-lemma}, we have a splitting $\mc{O}\otimes (f_1 U + f_2 U) \isom \mc{E}_1 \oplus \mc{O}$ such that $\im(M|_T) \subseteq \mc{E}_1$. 

  Consider the map
  $\wtilde M|_T \from (\mc{O} \otimes U) \oplus (\mc{O} \otimes U) \to \mc{O} \otimes (f_1 U + f_2 U)$
  defined by $$\wtilde M|_T(a\otimes \theta_1,b \otimes \theta_2)=a\otimes f_1 \theta_1 + b \otimes f_2 \theta_2.$$
  We obtain the matrix description of $\wtilde M|_T$ from the matrix description of $M|_T$ as follows. If $M|_T$ is represented by the matrix $A$ with coefficients $A_{i,j} = \lambda_{i,j} s + \mu_{i,j} t$, $i\leq \dim (f_1 U + f_2 U)$, $j\leq \dim U$, then $\wtilde M|_T$ is represented by a block matrix
  \[
  	B = \left(
  		\begin{array}{c|c}
  			A' & A'' \\
  		\end{array}
  	\right)
  \]
  with $A'_{i,j} = \lambda_{i,j}$ and $A''_{i,j} = \mu_{i,j}$.

  The property $\im(M|_T)\subseteq \mc{E}_1$ implies that after some row operations, the matrix $A$ has a zero row. By the construction of $\wtilde M|_T$, the same row operations lead to the matrix $B$ having a zero row, but this is a contradiction, since the map $\wtilde M|_T$ is surjective.
\end{proof}

\begin{remark}
	The general type candidate for $V_k$ takes the form
	$(b^{(k)}+1,\dotsc,b^{(k)}+1,b^{(k)},\dotsc,b^{(k)})$,
	where the number of entries is
	$r^{(k)} = \binom{n+k}{n} - \binom{n+k-d}{n}$
	and their sum $d^{(k)} = \binom{n+k-d}{n}$, while
	$b^{(k)} = \lfloor d^{(k)}/r^{(k)}\rfloor$.
	Note that the degrees of $d^{(k)}$ and $r^{(k)}$ as polynomials in $k$ are $n$ and $n-1$, respectively. Hence, $b^{(k)} \to \infty$ for $k \to \infty$.
\end{remark}

\begin{corollary} \label{no-more-than-ones}
	Let $t\in \Gr(2,H^0(\PP^n, \mc{O}(d)))$ be a line spanned by the polynomials $f_1,f_2$. Let $(\theta_j)$ be a monomial basis of $H^0(\PP^n, \mc{O}(k-d))$. Let $k$ be such that $b^{(k)}=0$, that is such that in the general type, only ones and zeroes appear. The bundle $V_{k,t}$ has general type if and only if
	$(f_1\theta_j,f_2\theta_j)_j$ is a linearly independent set in $H^0(\PP^n,\mc{O}(k))$.
\end{corollary}

\begin{proof}
  Since $b^{(k)}=0$, the type of $V_{k,t}$ is the general type if and only if it has $d^{(k)}$ many nonzero entries.
	By \Cref{number-zeroes}, this is the case if and only if $\dim\gen{f_1 \theta_j, f_2 \theta_j}_j = 2 d^{(k)}$.
\end{proof}

\begin{proposition} \label{nongeneral-type-shared-sections}
	Let $t\in \Gr(2,H^0(\PP^n, \mc{O}(d)))$ be a line spanned by the polynomials $f_1,f_2$, and let $k$ be such that $b^{(k)}=0$. The bundle $V_{k,t}$ has nongeneral type if and only if $\deg(\gcd(f_1,f_2)) \geq 2d-k$. In particular, if $b^{(k)}=0$ but $k>2d$ then the general type never occurs.
\end{proposition}

\begin{proof}
	By \Cref{no-more-than-ones}, the bundle $V_{k,t}$ has non-general type if and only if there exist linearly independent $g_1,g_2\in H^0(\PP^n,\mc{O}(k-d))$ such that $g_1f_1+g_2f_2 = 0$. Let $h \coloneqq \gcd(f_1,f_2)$ and $d'\coloneqq \deg h$.

	If $d' \geq 2d-k$ then $\deg (f_i/h) \leq k-d$ and we may take $g_1,g_2$ to be multiples of $f_1/h$ and $f_2/h$, respectively.

	On the other hand, given such $g_1$ and $g_2$, we have $f_1\divides g_2 f_2$, which implies $f_1/h \divides g_2$, hence $d-d'\leq k-d$.
\end{proof}

\begin{example} \label{no-big-types}
	For $n=2, d=2,$ and $k=3$, we have $d^{(k)}=3$ and $r^{(k)}=10$. We show that the only types of $V_k$ that occur are $(1_3, 0_7)$ and $(2_1,1_1,0_8)$. The first type occurs \eg for $f_1 = x_0^2, f_2=x_1^2$, and the second for $f_1=x_0^2, f_2=x_0x_1$. Assume that the type $(3_1,0_9)$ occurs for some $f_1,f_2 \in H^0(\PP^2,\mc{O}(2))$. By \Cref{number-zeroes} we then have $\dim\gen{f_1 x_j,f_2 x_j}_{j=0}^2 = 4$. Hence, we find $g_1,g_2,g'_1,g'_2\in H^0(\PP^2,\mc{O}(1))$ and two linearly independent equations
	\begin{align*}
	g_1f_1 + g_2f_2 &= 0 \\
	g'_1f_1 + g'_2f_2 &= 0,
	\end{align*}
	with both sets $(g_1,g_2), (g'_1,g'_2)$ linearly independent.
	From the first equation it follows that $f_1 = g_2 h$ and $f_2 = -g_1 h$, for some common linear factor $h$. Applying this to the second equation, we find $g'_1 g_2 = g'_2 g_1$, hence $g'_1 = \alpha g_1$ and $g'_2 = \alpha g_2$ for some scalar $\alpha$, a contradiction.
\end{example}

\begin{proposition} \label{no-big-types-general}
	Let $k=d+1$. No types of $V_k$ other than
	$(1,\dotsc,1,0,\dotsc,0)$ and $(2,1,\dotsc,1,0,\dotsc,0)$ occur. 
\end{proposition}

\begin{proof}
	The proof follows the lines of \Cref{no-big-types}. Assume that the type of $V_k$ at some line $(f_1,f_2)$ is other than the two above. Then the type has two more zero entries than the general type, corresponding to two equations of the form
	\begin{align*}
		g_1f_1 + g_2f_2 =& 0 \\
		g'_1 f_1 + g'_2 f_2 =& 0,
	\end{align*}
	with $g_i,g'_i\in H^0(\PP^n,\mc{O}(1))$. We use the irreducibility of the $g_i$ to produce a contradiction just like in the cited example, the only difference being that the common factor $h$ of $f_1$ and $f_2$ need not be linear.
\end{proof}