% \newcommand{\mco}{\multicolumn{1}{c}}
% A $K$-point $q$ of $\Gr(2,35)$ is given by a matrix of the form
% \[
% 	\begin{pmatrix}
% 		\lambda_0 & \cdots & \lambda_{34} \\
% 		\mu_0 & \cdots & \mu_{34}
% 	\end{pmatrix},
% \]
% with $\lambda_i,\mu_i \in K,$ up to elementary row operations. Restricting the sequence from \Cref{verlinde-exact-sequence} to the quartic defined by $q$, we see that the bundle $V_{5,q}$ is the cokernel of the matrix $A \in \Mat(56\times 4, \Gamma(\PP^1, \mc{O}(1))$, given as follows. Let $s,t$ denote the homogeneous coordinates of $\PP^1$ and let $I_d$ range over the tuples of the form $(i_0,\dotsc,i_3)$ with $\sum i_j = d.$ The $(I_5,j)$-th entry of $A$ is
% $s\lambda_{I_4} + t\mu_{I_4}$ if $x_jx^{I_4} = x^{I_5}$ and $0$ else.

% Because of the invariance under elementary row operation, the point $q$ can be defined by a matrix of the form
% \[
% 	\left(
% 		\begin{array}{c|cccccccc}
% 			\multirow{2}{*}{0} & 1 & \lambda_1 & \cdots & \lambda_k & 0 & \lambda_{k+1} & \cdots & \lambda_{k+l} \\
% 			& 0 & 0 & \cdots & 0 & 1 & \mu_1 & \cdots & \mu_l
% 		\end{array}
% 	\right).
% \]
% We hence start with the matrix $A$ having the form
% \[
% 	\left(
% 		\begin{array}{c|ccc}
% 			s & 0 & 0 & 0 \\
% 			\cline{1-4}
% 			\lambda_1 s & & & \\
% 			\vdots & & \ast & \\
% 			\lambda_k s & & & \\
% 			t & & & \\
% 			\cline{1-4}
% 			\multicolumn{1}{c}{} & \ast & &
% 		\end{array}
% 	\right),
% \]
% with $\ast$ for now not specified, but not arbitrary. 

% We may perform elementary row and column operations without changing $\coker(A).$ Using \Cref{rem:exact-sequence-exists}, the goal is to show that the matrix $A$ can be modified to take one of the following forms:

% \[
% 	\left(
% 		\begin{array}{cccc}
% 			s &   &   &   \\
% 			t & s &   &   \\
% 			  & t &   &   \\
		
% 			  &   & s &   \\
% 			  &   & t &   \\
			
% 			  &   &   & s \\
% 			  &   &   & t \\
% 			\hline
% 			 & \multicolumn{1}{c}{0} & \multicolumn{1}{c}{} & 
% 		\end{array}
% 	\right), \quad
% 	\left(
% 		\begin{array}{cccc}
% 			s &   &   &   \\
% 			t &   &   &   \\
% 			  & s &   &   \\
% 			  & t &   &   \\
% 			  &   & s &   \\
% 			  &   & t &   \\
% 			  &   &   & s \\
% 			  &   &   & t \\
% 			\hline
% 			 & \multicolumn{1}{c}{0} & \multicolumn{1}{c}{} & 
% 		\end{array}
% 	\right).
% \]

% We see that this is possible in \Cref{types-for-v-five}.
% \end{proof}

% \begin{computation} \label{types-for-v-five}
% 	[Vielleicht abstrakteres Argument hier]
% \end{computation}